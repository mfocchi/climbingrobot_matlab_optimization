We presented a  robotic solution for exploration and rescue 
missions in mountain environments such as canyons, lunar craters, etc. 
It could be employed in civil applications in scaling maintenance activity, e.g., to detach from the
mountain walls dangerous boulders,  needed to mitigate the hydro-geological risk.
%or loose and potentially unstable vegetation (a.k.a. scaling) or to apply landslide protection networks
%maintenance costs in scaling application
This would create additional market opportunities for a possible business.
Our platform combines the use of a rope of adjustable length with a jumping mechanism.
We validated the result of the optimization carried out using a simplified model 
with a fully-detailed model  simulation in Gazebo,
obtaining  $5\%$ discrepancy on a 16 $m$ jump.
% limitations
The main limitations of the actual work are the limited \textit{reachability} with 
the single anchor arrangement and the lack of descriptiveness (in terms of angular dynamics) of the simplified model.
This prevents to devise planning strategies that optimize also the orientation of the robot at the landing. 
To address the first issue, we plan to investigate a solution with two ropes with more than one anchor point to increase the reachable region on the wall;
while, for the second issue, we plan to release the point-mass approximation and considering the robot as a rigid body with
non-trivial mass geometry. In this case, the thrust action has to be combined with a way to control the rotation of the body 
and in order to ensure the proper alignment of the leg  to guarantee a safe landing.  
Last, we are working on the design of a landing controller to dissipate the excess of kinetic energy at landing, 
avoiding rebounces. All these steps are preparatory to the development of a working prototype.
%
Many important future directions have been opened by this research.
Whilst avoiding obstacles is a well-known problem for motion planning in horizontal terrain, 
the jump motion pattern determined by the ropes makes the motion-planning
problem non-standard, calling for new approaches.
Much work has to be done also in the area of motion control in order to
ensure that the planned trajectory is closely tracked during the flight. 

