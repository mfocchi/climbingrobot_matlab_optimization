% %%%%%%%%%%%%%%%%%%%%%%%%%%%%%%%%%%%%%%%%%%%%%%%%%%%%%%%%%%%%%%%%%%%%%%%%%%%%%%%
\documentclass[journal,letterpaper]{IEEEtran}
% include the list of acronyms, math commands and new commands used in this paper
% \usepackage[backend=bibtex,maxnames=2]{biblatex}
\usepackage[pdftex]{graphicx}
\usepackage[numbers]{natbib}
\bibliographystyle{IEEEtran}
\usepackage{booktabs}
\usepackage{moreverb}
%\usepackage{titlesec}
%\usepackage[titletoc,toc,title]{appendix}
\usepackage{url}
\usepackage{amsmath,amssymb}
\usepackage{multicol,lipsum}

\usepackage{bm}
\usepackage{subfig}
\usepackage{color, colortbl}
\usepackage[colorlinks,bookmarksopen,bookmarksnumbered,citecolor=red,urlcolor=red]{hyperref}


\usepackage{enumitem}

\hypersetup
{
	pdftitle = {Whole-body control for quadrupedal locomotion on challenging terrain},
	pdfauthor = {Shamel Fahmi},
	pdfsubject = {RA-L manuscript},
	pdfkeywords = {whole-body control, legged robots, challenging locomotion, and
	terrain mapping},
	pdftoolbar = true,
	colorlinks = true,
	linkcolor = black,
	citecolor = black,
	urlcolor = black,
}

\usepackage[usenames,dvipsnames]{xcolor}%\usepackage{xcolor,colortbl}
\definecolor{blue_iit}{RGB}{51,51,255}
\usepackage{algpseudocode}
\usepackage{algorithm}
\usepackage{glossaries}
\usepackage[tight]{units}
\usepackage[normalem]{ulem} % to strike out text, use: \sout{text}
\usepackage{cancel}
\definecolor{Gray}{gray}{0.9}

%\usepackage{cleveref}
%\crefname{figure}{Fig.}{Fig.}
%\crefname{equation}{Eq.}{Eq.}
%\AtBeginDocument{%
%  \renewcommand{\crefpairconjunction}{,}%% instead of " and\nobreakspace"
%  \renewcommand{\crefmiddleconjunction}{,}% instead of ", "
%  \renewcommand{\creflastconjunction}{,}% instead of " and\nobreakspace"
%}

\input{includes/acronyms.tex}
\input{includes/math_commands.tex}
\newcommand{\sref}[1]{Section~\ref{#1}}
%\newcommand{\eref}[1]{Eq.~(\ref{#1})}
\newcommand{\eref}[1]{(\ref{#1})}
\newcommand{\fref}[1]{Fig.~\ref{#1}}
\newcommand{\tref}[1]{Table~\ref{#1}}



%\newtheorem{Assumption}{Assumption}[section]
\newtheorem{assump}{Assumption}
\newtheorem{assumpB}{Assumption}
\renewcommand\theassump{1}
\renewcommand\theassumpB{2}
\newcommand{\assref}[1]{Assumption~\ref{#1}}



\newcommand{\MF}[1]{\textcolor{red}{\textbf{mfocchi}: #1}}
\newcommand{\LP}[1]{\textcolor{violet}{\textbf{lpalopoli}: #1}}
\newcommand{\ADP}[1]{\textcolor{blue}{\textbf{adelprete}: #1}}




\newcommand\BibTeX{{\rmfamily B\kern-.05em \textsc{i\kern-.025em b}\kern-.08em
T\kern-.1667em\lower.7ex\hbox{E}\kern-.125emX}}


\newcommand{\ie}{{i.e.},\ }
\newcommand{\eg}{{e.g.},\ }
\newcommand{\etal}{{\textit{et~al.}}\ }






%\usepackage[table]{xcolor}
%\definecolor{DarkGray}{RGB}{0.25,0.25,0.25}
%\definecolor{Gray}{RGB}{0.5,0.5,0.5}
%\definecolor{Red}{RGB}{1,0,0}
\definecolor{LightBlue}{RGB}{0.4,0.4,1}
\newcommand{\thickhline}{\noalign{\hrule height 0.8pt}}

\newcommand{\bmcolor}[1]{\textcolor{RoyalBlue}{\bm{#1}}}



\title{CLIO: a novel robotic solution for exploration and rescue missions in hostile mountain environments}

\author{Michele Focchi$^{1}$, Mohamed Bensaadallah$^{2}$, Marco Frego$^{3}$, Angelika Peer$^{3}$, Daniele Fontanelli$^{4}$, Andrea Del Prete$^{4}$, Luigi Palopoli$^{1}$} 
% \thanks{$^1$ The authors are with the Diaprtimento di Ingegneria and Scienza dell'Informazione (DISI), University of Trento. Email:  \href{mailto:name.surname@unitn.it}{name.surname@unitn.it}}
% \thanks{$^2$ The author is ... . Email: } \href{mailto:name.surname@...}{name.surname@...}}
 %\thanks{$^3$ The author are with University of Bolzano Email:  \href{mailto:name.surname@unibz.it}{name.surname@unibz.it}}
% \thanks{$^4$ The author are with Dipartimento di Ingengneria Industriale(DII), University of Trento Email:  \href{mailto:name.surname@unitn.it}{name.surname@unitn.it}}


\begin{document}
\maketitle
\thispagestyle{empty}
\pagestyle{empty}

\begin{abstract}%150-250 word abstract
\end{abstract}

\begin{IEEEkeywords}
Control, Planning, climbing robot, optimization,  
\end{IEEEkeywords}

\section{Introduction}\label{sec:introduction}
% introduction of the problem
Today, Robots are capable of climbing vertical walls replacing humans to do difficult tasks and preventing them from dangerous situations \cite{tag0}.The idea of developing climbing robots was expected for years but only until 1986, A.NISHI et al. were the first who designed a robot that can walk on a vertical wall in their seminal paper \cite{seminal_paper}. After that during the last three decades many prototypes (over 200) were proposed according to the desired purpose of the climbing robot e.g, inspection, cleaning, and maintenance of tall towers,.. etc. However, the designing structure requires solving two problems. one is the type of adhesion method that should be used to attach the robot to the wall. and second, the type of walking mechanism that should be used to climb the wall \cite{TAVAKOLI2015301}. 





   






\subsection{Related Work}


\section{Robot Description}\label{sec:}

\cite{tag0}

\section{Motion Planning}\label{sec:}




\section{Control}\label{sec:}




\section{Experiments}\label{sec:}





\section{Conclusions}
\label{sec:conclusion}

\small
\bibliographystyle{IEEEtran}
\bibliography{references/references}

\section{Acknowledgements}
The publication was created with the co-financing of the European Union FSE-REACT-EU, PON Research and Innovation 2014-2020 DM1062 / 2021.

\end{document}

