% %%%%%%%%%%%%%%%%%%%%%%%%%%%%%%%%%%%%%%%%%%%%%%%%%%%%%%%%%%%%%%%%%%%%%%%%%%%%%%%
\documentclass[journal,letterpaper]{IEEEtran}
% include the list of acronyms, math commands and new commands used in this paper
% \usepackage[backend=bibtex,maxnames=2]{biblatex}
\usepackage[pdftex]{graphicx}
\usepackage[numbers]{natbib}
\bibliographystyle{IEEEtran}
\usepackage{booktabs}
\usepackage{moreverb}
%\usepackage{titlesec}
%\usepackage[titletoc,toc,title]{appendix}
\usepackage{url}
\usepackage{amsmath,amssymb}
\usepackage{multicol,lipsum}

\usepackage{bm}
\usepackage{subfig}
\usepackage{color, colortbl}
\usepackage[colorlinks,bookmarksopen,bookmarksnumbered,citecolor=red,urlcolor=red]{hyperref}


\usepackage{enumitem}

\hypersetup
{
	pdftitle = {Whole-body control for quadrupedal locomotion on challenging terrain},
	pdfauthor = {Shamel Fahmi},
	pdfsubject = {RA-L manuscript},
	pdfkeywords = {whole-body control, legged robots, challenging locomotion, and
	terrain mapping},
	pdftoolbar = true,
	colorlinks = true,
	linkcolor = black,
	citecolor = black,
	urlcolor = black,
}

\usepackage[usenames,dvipsnames]{xcolor}%\usepackage{xcolor,colortbl}
\definecolor{blue_iit}{RGB}{51,51,255}
\usepackage{algpseudocode}
\usepackage{algorithm}
\usepackage{glossaries}
\usepackage[tight]{units}
\usepackage[normalem]{ulem} % to strike out text, use: \sout{text}
\usepackage{cancel}
\definecolor{Gray}{gray}{0.9}

%\usepackage{cleveref}
%\crefname{figure}{Fig.}{Fig.}
%\crefname{equation}{Eq.}{Eq.}
%\AtBeginDocument{%
%  \renewcommand{\crefpairconjunction}{,}%% instead of " and\nobreakspace"
%  \renewcommand{\crefmiddleconjunction}{,}% instead of ", "
%  \renewcommand{\creflastconjunction}{,}% instead of " and\nobreakspace"
%}

\input{includes/acronyms.tex}
\input{includes/math_commands.tex}
\newcommand{\sref}[1]{Section~\ref{#1}}
%\newcommand{\eref}[1]{Eq.~(\ref{#1})}
\newcommand{\eref}[1]{(\ref{#1})}
\newcommand{\fref}[1]{Fig.~\ref{#1}}
\newcommand{\tref}[1]{Table~\ref{#1}}



%\newtheorem{Assumption}{Assumption}[section]
\newtheorem{assump}{Assumption}
\newtheorem{assumpB}{Assumption}
\renewcommand\theassump{1}
\renewcommand\theassumpB{2}
\newcommand{\assref}[1]{Assumption~\ref{#1}}



\newcommand{\MF}[1]{\textcolor{red}{\textbf{mfocchi}: #1}}
\newcommand{\LP}[1]{\textcolor{violet}{\textbf{lpalopoli}: #1}}
\newcommand{\ADP}[1]{\textcolor{blue}{\textbf{adelprete}: #1}}




\newcommand\BibTeX{{\rmfamily B\kern-.05em \textsc{i\kern-.025em b}\kern-.08em
T\kern-.1667em\lower.7ex\hbox{E}\kern-.125emX}}


\newcommand{\ie}{{i.e.},\ }
\newcommand{\eg}{{e.g.},\ }
\newcommand{\etal}{{\textit{et~al.}}\ }






%\usepackage[table]{xcolor}
%\definecolor{DarkGray}{RGB}{0.25,0.25,0.25}
%\definecolor{Gray}{RGB}{0.5,0.5,0.5}
%\definecolor{Red}{RGB}{1,0,0}
\definecolor{LightBlue}{RGB}{0.4,0.4,1}
\newcommand{\thickhline}{\noalign{\hrule height 0.8pt}}

\newcommand{\bmcolor}[1]{\textcolor{RoyalBlue}{\bm{#1}}}



\title{CLIO: a novel robotic solution for exploration and rescue missions in hostile mountain environments}

\author{Michele Focchi$^{1}$, Mohamed Bensaadallah$^{2}$, Marco Frego$^{3}$, Angelika Peer$^{3}$, Daniele Fontanelli$^{4}$, Andrea Del Prete$^{4}$, Luigi Palopoli$^{1}$} 
% \thanks{$^1$ The authors are with the Diaprtimento di Ingegneria and Scienza dell'Informazione (DISI), University of Trento. Email:  \href{mailto:name.surname@unitn.it}{name.surname@unitn.it}}
% \thanks{$^2$ The author is ... . Email: } \href{mailto:name.surname@...}{name.surname@...}}
 %\thanks{$^3$ The author are with University of Bolzano Email:  \href{mailto:name.surname@unibz.it}{name.surname@unibz.it}}
% \thanks{$^4$ The author are with Dipartimento di Ingengneria Industriale(DII), University of Trento Email:  \href{mailto:name.surname@unitn.it}{name.surname@unitn.it}}


\begin{document}
\maketitle
\thispagestyle{empty}
\pagestyle{empty}

\begin{abstract}%150-250 word abstract
\end{abstract}

\begin{IEEEkeywords}
Control, Planning, climbing robot, optimization,  
\end{IEEEkeywords}

\section{Introduction}\label{sec:introduction}
%Strong background references addresses "what is already known" 
The idea of developing climbing robots was expected for years but only until 1986, A.NISHI et al. were the first who designed a robot that can walk on a vertical surface in their seminal paper \cite{seminal_paper}. In addition, many prototypes were proposed during the last thirty years to replace humans in difficult tasks and preventing them from dangerous situations. However, 
% Mention a Gap in knowledge
there are still other intelligent ways to climb walls not yet inspired from nature \cite{bio_inspired2015}.

% Introduction of the problem 
The research work done on climbing robots is classified in \cite{A.Hajeeretall} and can be summarized in \cite{Fang2022} based on both attachment, adhesion method that should be used to attach the robot to the wall and locomotion, the type of walking mechanism to climb the wall. which are considered as the two main problems to be solved in designing climbing robots. Other capabilities may be also desired: R1) ranging from carrying a payload, R2) be able to move fast (.e.g. in search and rescue missions).
 
% This coming part will be improved with the help of Michele!  
% limits wiht high inclinatoions
For the attachment technique, It is well known that there is a limit in the inclination a robot can address related to the 
value of the friction coefficient at the contact. that's why we suggest using a rope to attach our robot to the wall.
% advantages of using a rope (summarize a bit todo)
The advantage of using a the rope is to allow the contact
forces to better satisfy friction constraints (i.e., be more in the
middle of the friction cones) when walking on highly inclined
terrains. Indeed, in a slope with a very high
inclinations ) \cite{troabdlallah}, the robot eventually creates a tangential force on
the terrain that surpasses the friction force that is needed to
prevent slippage. A rope can introduce an external force to
compensate for gravity solving this issue and allowing the
robot to walk on ideally any terrain inclination (even vertical).
%advantages of using a rope
An additional advantage of using a rope in legged robots \cite{} is
that the robot can keep a more kinematically advantageous configuration (i.e. far from kinematic limits) without the need to lean back or forth to keep stability. 
%advantages of using a rope
Furthermore, with respect to most common climbing robots that uses sticky pads \cite{Kim2008}, \cite{Riskin2009} and employ locomotion gaits to climb up/down a wall, the aid of a rope like the novel bio-inspired dragline locomotion in \cite{Wang2014} can dramatically increase the locomotion speed (R2) by a winding/releasing mechanism, therefore being a preferable solution in applications that require a prompt intervention like rescue missions. 

% we can reach the target with jumps 
For the locomotion mechanism, we suggest using the jumping as a moving method on the wall. Indeed, rather than slowly taking steps to move, the robot can take one or multiple jumps to reach the target on the wall. The jumping locomotion is similar to the Salto-1P jumping robot shown in \cite{Haldane2017} but the difference (key feature) of climbing robots relies in the ability to address terrains of high inclination (up to vertical or beyond). For each jump it exploits the winding/releasing  mechanism to behave like a variable length  pendulum ( see Fig. \ref{fig:}). 

% advantages od jumping
Additionally, the amount of payload that can be carried by sticky/adhesive based climbing robots is very limited due to the leg actuation limits and tangential component of the adhesive force that is limited.
While using a rope, being it inextensible (on a certain extent) the amount of payload the  robot is able to carry is much higher (requirement R1) and is only limited by the maximum force deliverable by the windind mehcanism (e.g. an hoist), which using a gearbox can be designed to be very high. 

% last advantage of jumping
Finally, a rope ensures an inherent safety (the robot cannot fall) and efficiency when rapelling down (i.e. exploiting the effect of gravity to unwind the rope. 
 
 
%difficulties due to the dynamics
However, having the robot attached to a rope poses many challenges: one is the under-actuation (the robot is not able to control the position of its center of mass while not in contact) and the other is that a rope represents a 
unilateral constraint (can only pull and not push), which further complicates the already hybrid dynamics of this class of robots. 

% difficulties in the jump
Taking a jump with a rope that can be released, the resulting motion of the robot (and so the possibility to reach the target) will depend on two factors: 1) the initial impulse that the robot exerts  on the wall to lift-off, 2) the time-law the rope is released/re-winded.

% optimization for plannig 
Therefore, a planning strategy for these kind of motions, to be successful, 
it should take into account both the concurrent action of these two factors  
and under-actuation of the robot. Additionally, 
the constrains posed by the rope, the actuator limits and the contact interaction (i.e. friction) should be fulfilled.

Numerical optimization is an  attractive solution for this planning problem 
because it allows to jointly cope with the two factors that influence  the motion in an optimal fashion
while ensuring that constraints  are satisfied in a planning horizon resulting in a feasible trajectory. 
 
Casting the locomotion planning as an optimization
problem allows one to represent represent high level tasks and system dynamics
using cost functions and constraints. 
In this framework different gaols can also be pursued. For instance, energy consumption or hoist effort can be minimized to  reduce battery consumption and extend mission duration or jump time can be minimized to achieve the target in the shortest time in case of   emergencies or  rescue missions. 


\subsection{Proposed Approach and Contribution}
In this work, we present a novel robotic platform called CLIO that is able to 
reach desired targets on a  vertical wall. We  propose  planning approaches 
based on numerical optimization to solve the jumping problem employing a simplified model of the dynamics.
To summarize, the contributions of the paper are:
\begin{itemize}
	\item a novel design of a jumping robot platform CLIO 
	\item two planning approaches, one based on iso-energetic curves  and one on optimal control to generate a desired jump motion, comparing pros and cons of each approach.
	\item simulations and experiments to demonstrate the effectiveness of the proposed approaches in a realistic (gazebo) simulation with the full dynamics of the robot.
\end{itemize}


\subsection{Outline}

The paper is organized as follows: Section II gives an
overview of the robot ....
on III describes the optimization
problem with the LIP model and how it is used to compute
CoM position velocity and contact force references. Simula-
tions and experiments with Aliengo robot are illustrated in
Section IV. Finally, we draw the conclusions in Section V.

\section{Robot Description}\label{sec:}

\cite{tag0}

\section{Motion Planning}\label{sec:}




\section{Control}\label{sec:}




\section{Experiments}\label{sec:}





\section{Conclusions}
\label{sec:conclusion}
% future works 
extend optimization to the full dynamics considering angular dynamics 
multiple jumps 
design of a non linear controller to more efficiently track the planned trajectory
experiments with the real platform 

\small
\bibliographystyle{IEEEtran}
\bibliography{references/references}

\section{Acknowledgements}
The publication was created with the co-financing of the European Union FSE-REACT-EU, PON Research and Innovation 2014-2020 DM1062 / 2021.

\end{document}

