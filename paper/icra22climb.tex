% %%%%%%%%%%%%%%%%%%%%%%%%%%%%%%%%%%%%%%%%%%%%%%%%%%%%%%%%%%%%%%%%%%%%%%%%%%%%%%%
\documentclass[journal,letterpaper]{IEEEtran}
% include the list of acronyms, math commands and new commands used in this paper
% \usepackage[backend=bibtex,maxnames=2]{biblatex}
\usepackage[pdftex]{graphicx}
\usepackage[numbers]{natbib}
\bibliographystyle{IEEEtran}
\usepackage{booktabs}
\usepackage{moreverb}
%\usepackage{titlesec}
%\usepackage[titletoc,toc,title]{appendix}
\usepackage{url}
\usepackage{amsmath,amssymb}
\usepackage{multicol,lipsum}

\usepackage{bm}
\usepackage{subfig}
\usepackage{color, colortbl}
\usepackage[colorlinks,bookmarksopen,bookmarksnumbered,citecolor=red,urlcolor=red]{hyperref}


\usepackage{enumitem}

\hypersetup
{
	pdftitle = {Whole-body control for quadrupedal locomotion on challenging terrain},
	pdfauthor = {Shamel Fahmi},
	pdfsubject = {RA-L manuscript},
	pdfkeywords = {whole-body control, legged robots, challenging locomotion, and
	terrain mapping},
	pdftoolbar = true,
	colorlinks = true,
	linkcolor = black,
	citecolor = black,
	urlcolor = black,
}

\usepackage[usenames,dvipsnames]{xcolor}%\usepackage{xcolor,colortbl}
\definecolor{blue_iit}{RGB}{51,51,255}
\usepackage{algpseudocode}
\usepackage{algorithm}
\usepackage{glossaries}
\usepackage[tight]{units}
\usepackage[normalem]{ulem} % to strike out text, use: \sout{text}
\usepackage{cancel}
\definecolor{Gray}{gray}{0.9}

%\usepackage{cleveref}
%\crefname{figure}{Fig.}{Fig.}
%\crefname{equation}{Eq.}{Eq.}
%\AtBeginDocument{%
%  \renewcommand{\crefpairconjunction}{,}%% instead of " and\nobreakspace"
%  \renewcommand{\crefmiddleconjunction}{,}% instead of ", "
%  \renewcommand{\creflastconjunction}{,}% instead of " and\nobreakspace"
%}

\input{includes/acronyms.tex}
\input{includes/math_commands.tex}
\newcommand{\sref}[1]{Section~\ref{#1}}
%\newcommand{\eref}[1]{Eq.~(\ref{#1})}
\newcommand{\eref}[1]{(\ref{#1})}
\newcommand{\fref}[1]{Fig.~\ref{#1}}
\newcommand{\tref}[1]{Table~\ref{#1}}



%\newtheorem{Assumption}{Assumption}[section]
\newtheorem{assump}{Assumption}
\newtheorem{assumpB}{Assumption}
\renewcommand\theassump{1}
\renewcommand\theassumpB{2}
\newcommand{\assref}[1]{Assumption~\ref{#1}}



\newcommand{\MF}[1]{\textcolor{red}{\textbf{mfocchi}: #1}}
\newcommand{\LP}[1]{\textcolor{violet}{\textbf{lpalopoli}: #1}}
\newcommand{\ADP}[1]{\textcolor{blue}{\textbf{adelprete}: #1}}




\newcommand\BibTeX{{\rmfamily B\kern-.05em \textsc{i\kern-.025em b}\kern-.08em
T\kern-.1667em\lower.7ex\hbox{E}\kern-.125emX}}


\newcommand{\ie}{{i.e.},\ }
\newcommand{\eg}{{e.g.},\ }
\newcommand{\etal}{{\textit{et~al.}}\ }






%\usepackage[table]{xcolor}
%\definecolor{DarkGray}{RGB}{0.25,0.25,0.25}
%\definecolor{Gray}{RGB}{0.5,0.5,0.5}
%\definecolor{Red}{RGB}{1,0,0}
\definecolor{LightBlue}{RGB}{0.4,0.4,1}
\newcommand{\thickhline}{\noalign{\hrule height 0.8pt}}

\newcommand{\bmcolor}[1]{\textcolor{RoyalBlue}{\bm{#1}}}



\title{CLIO: a novel robotic solution for exploration and rescue missions in hostile mountain environments}

\author{Michele Focchi$^{1}$, Mohamed Bensaadallah$^{2}$, Marco Frego$^{3}$, Angelika Peer$^{3}$, Daniele Fontanelli$^{4}$, Andrea Del Prete$^{4}$, Luigi Palopoli$^{1}$} 
% \thanks{$^1$ The authors are with the Diaprtimento di Ingegneria and Scienza dell'Informazione (DISI), University of Trento. Email:  \href{mailto:name.surname@unitn.it}{name.surname@unitn.it}}
% \thanks{$^2$ The author is ... . Email: } \href{mailto:name.surname@...}{name.surname@...}}
%\thanks{$^3$ The author are with University of Bolzano Email:  \href{mailto:name.surname@unibz.it}{name.surname@unibz.it}}
% \thanks{$^4$ The author are with Dipartimento di Ingengneria Industriale(DII), University of Trento Email:  \href{mailto:name.surname@unitn.it}{name.surname@unitn.it}}


\begin{document}
\maketitle
\thispagestyle{empty}
\pagestyle{empty}

\begin{abstract}%150-250 word abstract
\end{abstract}

\begin{IEEEkeywords}
	Control, Planning, climbing robots, optimization,  
\end{IEEEkeywords}

\section{Introduction}\label{sec:introduction}
%Strong background references addresses "what is already known" 
The idea of developing climbing robots was expected for years but only until 1986, A.NISHI et al. were the first who designed a robot that can walk on a vertical surface in their seminal paper \cite{seminal_paper}. During the last thirty years, many prototypes of climbing robots were proposed for specific applications to replace humans in difficult tasks and preventing them from dangerous situations. However, % Mention a Gap in knowledge
there are still other combinations of intelligent ways not yet discussed \cite{bio_inspired2015}.
%Introduction of the problem 
The research work done so far on climbing robots is classified in \cite{A.Hajeeretall2020,Schmidt2013}, and can be summarized in \cite{Fang2022} based on both attachment, adhesion method that should be used to attach the robot to the wall and locomotion, type of the walking mechanism which are considered as the two main problems to be solved in designing climbing robots. Other requirements may also be necessary depending on the task such as: R1) less energy consumption, R2) ranging from carrying a payload, R3) be able to move fast e.g. in inspection \cite{Song2008,Dong2011} and rescue missions. 

%Providing our solution for first problem: attachment technique 
To solve the mentioned problems: Firstly, we propose using a rope as an adhesion mechanism for several reasons. 
% Advantages of using a rope : limits wiht high inclinatoions
One, it is well known that there is a limit in the inclination a mobile robot can address related to the value of the friction coefficient at the contact. So, using a rope will allows the contact forces to better satisfy the friction constraints (i.e., be more in the middle of the friction cones) when walking on highly inclined terrains. Indeed, in a slope with a very high inclinations)\cite{troabdlallah}, the robot eventually creates a tangential force on the terrain that surpasses the friction force that is needed to prevent slippage and falling. A rope can introduce an external force to compensate for gravity solving this issue and allowing the robot to walk on ideally any terrain inclination (even vertical).
%Advantages of using a rope
In addition, using a rope in legged robots \cite{} can keep more kinematically advantageous configuration (i.e. far from kinematic limits) without the need to lean back or forth to keep stability. In other words, the rope will ensure inherent safety from falling and efficiency when rappelling down (i.e. exploiting the effect of gravity to unwind the rope.)  
%Advantages of a rope
Furthermore, with respect to most common climbing robots that uses sticky pads \cite{Kim2008}, \cite{Riskin2009} and employ locomotion gaits to climb up/down a wall, the aid of a rope like the novel bio-inspired dragline locomotion in \cite{Wang2014} can dramatically increase the locomotion speed (R3) by a winding/releasing mechanism, therefore being a preferable solution in applications that require a prompt intervention like rescue missions. 
%Last advantage of rope
Finally, the amount of payload that can be carried by sticky/adhesive based climbing robots is very limited due to the leg actuation limits and tangential component of the adhesive force that is limited. While using a rope, being it inextensible (on a certain extent) the amount of payload the  robot is able to carry is much higher (requirement R2) and is only limited by the maximum force deliverable by the winding mechanism (e.g. an hoist), which using a gearbox can be designed to be very high. 
%Difficulties due to the dynamics when using a rope
However, having the robot attached to a rope poses many challenges: one is the under-actuation (the robot is not able to control the position of its center of mass while not in contact) and the other is that a rope represents a unilateral constraint (can only pull and not push), which further complicates the already hybrid dynamics of this class of robots. 

%Providing a solution for second problem: locomotion technique, we can reach the target with jumps 
Secondly, we suggest using the jumping as a moving method. Indeed, rather than slowly taking steps to move, the robot can take one or multiple jumps to reach the target on the wall. The jumping locomotion is similar to the Salto-1P jumping robot shown in \cite{Haldane2017} but the key feature of jumping with rope relies in the ability to address terrains of high inclination (up to vertical or beyond). For each jump, it exploits the winding/releasing mechanism to behave like a variable length pendulum (see Fig. \ref{fig:}). 
%Difficulties in the jumping technique
Taking a jump with a rope that can be released, the resulting motion of the robot (and so the possibility to reach the target) will depend on two factors: 1) the initial impulse that the robot exerts on the wall to lift-off, 2) the time-law the rope is released/re-winded.
%Optimization for plannig 
Therefore, a planning strategy for these kind of motions, to be successful, it should take into account both the concurrent action of these two factors and under-actuation of the robot. Also, consider the constrains posed by the rope, the actuator limits and the contact interaction (i.e. friction) should be fulfilled.

Numerical optimization is an attractive solution for this planning problem because it allows to jointly cope with the two factors that influence the motion in an optimal fashion while ensuring that the constraints are satisfied in a planning horizon resulting in a feasible trajectory. Casting the locomotion planning as an optimization problem allows one to represent high level tasks and system dynamics using cost functions and constraints. 
In this framework different goals can also be pursued. For instance, minimize energy consumption or hoist effort (requirement R1), jump time can be also minimized to achieve the target in a shortest time in case of emergencies or rescue missions and extend mission duration. 


\subsection{Proposed Approach and Contribution}
In this work, we present a novel robotic platform called CLIO that is able to 
reach desired targets on a vertical wall. We propose planning approaches 
based on numerical optimization to solve the jumping problem employing a simplified model of the dynamics.
To summarize, the contributions of the paper are:
\begin{itemize}
	\item a novel design of a jumping robot platform CLIO 
	\item two planning approaches, one based on iso-energetic curves  and one on optimal control to generate a desired jump motion, comparing pros and cons of each approach.
	\item simulations and experiments to demonstrate the effectiveness of the proposed approaches in a realistic (gazebo) simulation with the full dynamics of the robot.
\end{itemize}


\subsection{Outline}

The paper is organized as follows: Section II gives an
overview of the robot ....
on III describes the optimization
problem with the LIP model and how it is used to compute
CoM position velocity and contact force references. Simula-
tions and experiments with Aliengo robot are illustrated in
Section IV. Finally, we draw the conclusions in Section V.

\section{Robot Description}\label{sec:}

\subsection{Simplified Model}
We can define a simplified mathematical model by making the following assumptions: 1. The mass is entirely concentrated in the body attached to the wire, 2. We can unwind and rewind the wire but it is rigid and remains completely elongated (i.e., it cannot make bends), 3. The rewinder can pull the wire when winding it but cannot push it (i.e., it can only act as a brake during the unwinding phase).
A geometric sketch of the system is shown in Figure \ref{fig:sp}. where the polar angle is $\theta$ [rad], the Azimuth angle $\phi$ [rad], and $l$ [m] is the length of the rigid rope. 

\begin{figure}
\includegraphics[width=\columnwidth]{figs/spherical_pendulum}
\caption{Logical scheme for the simplified model}
\label{fig:sp}
\end{figure}
Our input variables to the system are: 1. the rewinder pull action or
braking action $\mathbf{F}_r$ oriented along the wire, 2. an impulsive
push force $\mathbf{F}_u$ that the robot can generate when it is
attached to the mountain wall. 

Let $e$ be a frame attached to the mass
and with axis oriented along the $x$ axis, and let $w$ be the world
frame attached to the suspension point of the pendulum.  If we
consider the homogeneous transformation linking $e$ to $w$, we can
write:
	\[
	T_e^w = \begin{bmatrix} R_z(\phi) &\begin{matrix}0\\0\\0 \end{matrix}  \\ \begin{matrix} 0 &0 &0 \end{matrix} & 1\end{bmatrix}  \begin{bmatrix} R_y(\pi/2 - \theta) &\begin{matrix}0\\0\\0 \end{matrix}  \\ \begin{matrix} 0 &0 &0 \end{matrix} & 1\end{bmatrix}  \begin{bmatrix} I_{3,3} &\begin{matrix}0\\0\\0 \end{matrix}  \\ \begin{matrix} L &0 &0 \end{matrix} & 1\end{bmatrix}  
	\]
where $R_z(\phi)$ is the rotation matrix of $\phi$ around $z$ , and $R_y(\alpha)$ is the rotation matrix of $\alpha$ around $y$. 
Simplifying $T_e^w$, gives :
	\[
	T_e^w = \begin{bmatrix} c_\phi s_\theta & -s_\phi & c_\phi c_\theta & l c_\phi s_\theta\\
		s_\phi s_\theta & c_\phi & c_\theta s_\phi & l s_\phi s_\theta\\
		-c_\theta & 0 & s_\theta &-l c_\theta\\
		0 & 0 &0 &1\end{bmatrix} 
	\]		
where $c_x$ is a shorthand for $\cos x$ , and $s_x$ is a shorthand for $\sin x$.
The position $\mathbf{p}$ of the mass can be extracted from the transformation matrix $T_e^w$ as : 
	\begin{align}  
		\mathbf{p} = \begin{bmatrix} x\\ y\\ z \end{bmatrix} =
		\begin{bmatrix}   
            l s_\theta c_\phi\\
			l s_\theta s_\phi\\
			-l c_\theta \end{bmatrix} \label{eq1}
    \end{align}
%From equation \ref{eq1}, we can compute the velocities and the kinetic energy:   
From equation \eqref{eq1}, the velocities along Cartesian axes are    
\begin{align*}
	\dot{x} &= 
	l c_\theta c_\phi \dot{\theta} - l s_\theta s_\phi \dot{\phi} + s_\theta c_\phi \dot{l} \\
	\dot{y} &= l c_\theta s_\phi \dot{\theta} + l s_\theta c_\phi \dot{\phi} + s_\theta s_\phi \dot{l} \\
	\dot{z} &= l s_\theta  \dot{\theta} - c_\theta \dot{l} 
\end{align*}
Next, the velocity squared of the mass 
\begin{align*}
	v^2 &= \dot{x}^2 + \dot{y}^2 + \dot{z}^2 =  l^2 \dot{\theta}^2 + l^2 s_\theta^2 \dot{\phi}^2 + \dot{l}^2 		
%	&=  l^2 \dot{\theta}^2 + l^2 s_\theta^2 \dot{\phi}^2 + \dot{l}^2\\
\end{align*}
Thus, the kinetic energy $T$ and potential energy $V$  
\begin{align*}
	T &=\frac{ m }{2} v^2 =  \frac{m}{2} l^2 \left(\dot{\theta}^2 + s_\theta^2 \dot{\phi}^2 \right) + \frac{m }{2} \dot{l}^2  \\
    V &= mgz = -mgl c_\theta   \label{eq3}
\end{align*}
which leading to the Lagrangian described as
\begin{equation}
	L = T - V = \frac{m}{2} l^2 \left(\dot{\theta}^2 + s_\theta^2 \dot{\phi}^2\right) + \frac{m}{2}\dot{l}^2 + mgl c_\theta
\end{equation}
The dynamics of the system will be obtained using the Euler-Lagrange equation:
\[
\frac{d}{dt}\frac{\partial L}{\partial \dot{q}_i} - \frac{\partial L}{\partial q_i} = Q_i^p ,
\]
where $Q_i^p = (\mathbf{F}_r + \mathbf{F}_u) \cdot \frac{\partial \mathbf{p}}{\partial q_i}$ is the generalized force, 	
	$\mathbf{F}_r$ is oriented along the rope, and $\mathbf{F}_u$ is generated so that it does not have any component along the rope.
	\begin{equation}
	\begin{aligned}
		\mathbf{F}_r  &= F_r \begin{bmatrix}
			c_\phi s_\theta\\
			s_\phi s_\theta\\
			-c_\theta
		\end{bmatrix}^T = F_r \mathbf{f}_r \\	
		\mathbf{F}_u &= F_{u,t}\begin{bmatrix}
			-s_\phi\\
			c_\phi\\
			0
		\end{bmatrix}^T + F_{u,n} \begin{bmatrix}
			c_\phi c_\theta \\
			s_\phi c_\theta\\
			s_\theta
		\end{bmatrix}^T &= F_{u,t} \mathbf{f}_{u,t} + F_{u,n} \mathbf{f}_{u,n}    	 
	\end{aligned}
	\label{eq:forces}
	\end{equation}
	Let the generalized coordinates $q_i$ be chosen as $q_1 = \theta$, $q_2 = \phi$ and $q_3 = l$ . The corresponding velocities are 
	\begin{align*}
		\frac{\partial \mathbf{p}}{\partial q_1} = \frac{\partial \mathbf{p}}{\partial \theta} &= \begin{bmatrix}
			l c_\phi c_\theta\\
			l s_\phi c_\theta\\
			l s_\theta
		\end{bmatrix},  \quad \text{and} \quad 
		\frac{\partial \mathbf{p}}{\partial q_2} = \frac{\partial \mathbf{p}}{\partial \phi} &= \begin{bmatrix}
			- l s_\phi s_\theta\\
			l c_\phi s_\theta\\
			0
		\end{bmatrix}, \\
		\frac{\partial \mathbf{p}}{\partial q_3} = \frac{\partial \mathbf{p}}{\partial l} &= \begin{bmatrix}
			c_\phi s_\theta\\
			s_\phi s_\theta\\
			-c_\theta
		\end{bmatrix} .
	\end{align*}
	Hence,
\begin{align*}
	Q_1^p &= \left(\mathbf{F}_r + \mathbf{F}_u\right)   \frac{\partial \mathbf{p}}{\partial q_1} 
	%% 		&= F_r \mathbf{f}_r \cdot  \frac{\partial \mathbf{p}}{\partial \theta} + \\
	%% 		&+ F_{u,t} \mathbf{f}_{u,t} \cdot  \frac{\partial \mathbf{p}}{\partial \theta} + \\
	%% 		&+ F_{u,n} \mathbf{f}_{u,n} \cdot  \frac{\partial \mathbf{p}}{\partial \theta}  =  \\
	%% 		&= F_{u,n} \mathbf{f}_{u,n} \cdot  \frac{\partial \mathbf{p}}{\partial \theta} = \\
	= F_{u,n} l , \\
	% 	\end{align*}
% 	\begin{align*}
	Q_2^p &= \left(\mathbf{F}_r + \mathbf{F}_u\right)   \frac{\partial \mathbf{p}}{\partial q_2}
	%% 		&= F_r \mathbf{f}_r \cdot  \frac{\partial \mathbf{p}}{\partial \phi} + \\
	%% 		&+ F_{u,t} \mathbf{f}_{u,t} \cdot  \frac{\partial \mathbf{p}}{\partial \phi} + \\
	%% 		&+ F_{u,n} \mathbf{f}_{u,n} \cdot  \frac{\partial \mathbf{p}}{\partial \phi}   = \\
	%% 		&= F_{u,t} \mathbf{f}_{u,t} \cdot  \frac{\partial \mathbf{p}}{\partial \phi} =\\
	= F_{u,t} l s_\theta,\\
	%% 	\end{align*}
%% 	\begin{align*}
	Q_3^p &= \left(\mathbf{F}_r + \mathbf{F}_u\right)   \frac{\partial \mathbf{p}}{\partial q_3}
	%% 		&= F_r \mathbf{f}_r \cdot  \frac{\partial \mathbf{p}}{\partial l} + \\
	%% 		&+ F_{u,t} \mathbf{f}_{u,t} \cdot  \frac{\partial \mathbf{p}}{\partial l} + \\
	%% 		&+ F_{u,n} \mathbf{f}_{u,n} \cdot  \frac{\partial \mathbf{p}}{\partial l}   = \\
	%% 		&= F_r \mathbf{f}_r \cdot  \frac{\partial \mathbf{p}}{\partial l} =\\
	= F_{r} .
\end{align*}
With the defined generalized coordinates, The Euler-Lagrange equations gives us : 	
%% 	The first Euler-Lagrange equation yields
 	\begin{align*}
%		&\frac{d}{dt}\left(\frac{\partial L}{\partial \dot{\theta}} \right) - \frac{\partial L}{\partial \theta} = Q_1^p  \rightarrow  \\
%%		& \frac{d}{dt}\left(m l^2 \dot{\theta}\right) - (m l^2 s_\theta c_\theta \dot{\phi}^2 - m g l s_\theta) = F_{u,n}  l \\ 
 		  &m l^2 \ddot{\theta} + 2 m l \dot{\theta} \dot{l} - m l^2 s_\theta c_\theta \dot{\phi}^2 + m g l s_\theta = F_{u,n} . l\\
%% 		& \ddot{\theta} + \frac{2}{l} \dot{\theta} \dot{l} - c_\theta s_\theta \dot{\phi}^2 + \frac{g}{l}  s_\theta =  \frac{F_{u,n}  }{ml}    
%% 	\end{align*}
%% 	The second Euler-Lagrange equation yields
%% 	\begin{align*}
%% 		&\frac{d}{dt}\left(\frac{\partial L}{\partial \dot{\phi}} \right) - \frac{\partial L}{\partial \phi} = Q_2^p    \\
%% 		&\frac{d}{dt}\left(m l^2 s_\theta^2 \dot{\phi}\right) = F_{u,t}  l  s_\theta\\
 		 &\ddot{\phi} ml^2 s^2_\theta + 2 m l \dot{l} s_\theta^2 \dot{\phi} + 2 m l^2 s_\theta c_\theta \dot{\theta} \dot{\phi}  = F_{u,t}  l  s_\theta\\
%%		&\ddot{\phi} + 2 \frac{c_\theta}{s_\theta} \dot{\theta} \dot{\phi} + \frac{2}{l} \dot{\phi} \dot{l} = \frac{F_{u,t}}{ml s_\theta}
%% 	\end{align*}
%% 	The third Euler-Lagrange equation is as follows:
%% 	\begin{align*}
%% 		&\frac{d}{dt}\left(\frac{\partial L}{\partial \dot{l}} \right) - \frac{\partial L}{\partial l} = Q_3^p    \\
%% 		&\frac{d}{dt}\left(m\dot{l}\right) - (m l \dot{\theta}^2 + m l s^2_\theta \dot{\phi}^2 + m g c_\theta) = F_r ,
 		&m \ddot{l} - m l \dot{\theta}^2 - m l s^2_\theta \dot{\phi}^2 - m g c_\theta = F_r 
%% 		&\ddot{l} - l \dot{\theta}^2 - l s^2_\theta \dot{\phi}^2 - g c_\theta = \frac{F_r}{m} 
 	\end{align*}
           %%
        Overall, the non-linear dynamic equations of the system 
%        Overall, the nonlinear equations of the system are:
        \begin{equation}
        \begin{aligned}
		&\ddot{\theta} + \frac{2}{l} \dot{\theta} \dot{l} - c_\theta s_\theta  \dot{\phi}^2 + \frac{g}{l}  s_\theta =  \frac{1}{ml}F_{u,n} \\
		&\ddot{\phi} + 2 \frac{c_\theta}{s_\theta} \dot{\theta} \dot{\phi} + \frac{2}{l} \dot{\phi} \dot{l} = \frac{1}{ml s_\theta} F_{u,t} \\
		&\ddot{l} - l \dot{\theta}^2 - l s^2_\theta \dot{\phi}^2 - g c_\theta = \frac{1}{m} F_r .
	\end{aligned}
        \label{eq:nonlinearDyn}
        \end{equation}
        Clearly, in this derivation of the model, we have heavily
        relied on the point-mass nature of the body.  A possible issue
        could be the rotational dynamics of the body around its axis,
        which could waste energy generating unnecessary motions and
        impede the landing phase. This issue will be part of our
        future work. However, as discussed next, under reasonable
        assumptions (e.g., thrusting force oriented in the direction
        of the center of mass), the results based on the simplified
        model can be applied to a realistic system with a good approximation.

\subsection{Robot motion: problems and solution overview}
The problem of moving CLIO between two given configurations is not an
easy task.  First of all, even the simplified dynamics in
Equation~\eqref{eq:nonlinearDyn} is highly nonlinear.  Second, 
the configurations the robot moves between can be in general two 
distant points of the workspace. Therefore, it is not possible to
use the linearized dynamics.  Third, one of our
actuators, the thrusting force $\mathbf{F}_u$, has an impulsive nature
and operates at discrete time instants. Therefore, it is not possible
to use this part of the actuation in any feedback control scheme.
Finally, the way the tangential component $F_{u,t}$ is generated is by
using friction. Therefore, the two components $F_{u,n}$ and $F_{u,t}$ are linked
by a nonlinear constraints expressing the friction cone. Because the friction cone
depends on the specific point or area where the robot is pointing its leg,
this mechanism is not totally reliable (i.e., there can be significant deviation
between the generated and the exacted values of $F_{u,t}$).

In view of this complexity, we propose an approach based on two steps:
1. first a motion strategy is planned prior to starting the motion,
2. after the take of phase a motion controller operates on
$\mathbf{F}_r$ in a feedback control scheme to fix small deviation and
to secure that the robot lands close enough to the expected position.


\cite{tag0}

\section{Motion Planning}\label{sec:}




\section{Control}\label{sec:}




\section{Experiments}\label{sec:}





\section{Conclusions}
\label{sec:conclusion}
% future works 
extend optimization to the full dynamics considering angular dynamics 
multiple jumps 
design of a non linear controller to more efficiently track the planned trajectory
experiments with the real platform 

\small
\bibliographystyle{IEEEtran}
\bibliography{references/references}

\section{Acknowledgements}
The publication was created with the co-financing of the European Union FSE-REACT-EU, PON Research and Innovation 2014-2020 DM1062 / 2021.

\end{document}

