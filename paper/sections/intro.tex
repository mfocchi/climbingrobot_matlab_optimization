
% Strong background references addresses "what is already known" 
The idea of developing climbing robots was expected for years but only in 1986, 
Nishi et al. presented the design of a robot that can walk on a vertical surface
in their seminal paper \cite{Nishi1986}. During the last thirty years, many prototypes
of climbing robots were proposed for specific applications to replace humans 
in dangerous and difficult tasks including, inspection in nuclear power plants \cite{Briones1994}, 
building cleaning \cite{Nansai2016}, bridges maintenance \cite{Wang2016},  
search and rescue missions \cite{Eich2008}. However, % Mention a Gap in knowledge
there are still other intelligent ways to climb walls not yet studied or reported 
\cite{bio_inspired2015}. %Introduction of the problem 

The research work done so far on climbing robots is reviewed and classified in 
\cite{Fang2022} and \cite{A.Hajeeretall2020} respectively, according to the adhesion 
method \cite{longo2008}, used to attach the robot to the wall and locomotion \cite{Chu2010}, 
type of the walking mechanism. These are considered as the two main problems to be solved in 
designing wall-climbing robots. Other requirements may also be requested depending on the 
desired task, namely: R1) to move safely, and R2) fast in emergency situations, R3) ability 
to carry a payload, and R4) avoiding obstacles, .. etc. 
%Providing our solution for first problem: attachment technique 

In this paper, we will discuss using the rope/wire as an attachment method and 
jumping as an moving technique to solve the mentioned problems of climbing robots to reach 
a desired targets on the vertical wall, overcoming obstacles.

%% citing the work used the rope as an attachment technique + first and main advantage of using a rope   
Firstly, humans inspired the technique of using a rope mainly for safety measures 
to support the weight of persons during facade cleaning, rescue, fire fighting, ..etc \cite{Nansai2016}.
unlike legged robots which needs to lean back and forth to keep stability, the rope 
ensures inherent safety (R1) from falling because the robot is always attached to wall.

%% second advantage of using a rope 
In addition, mobile robots in a slope with very high inclinations \cite{Abdalla12020}, creates a tangential 
force on the terrain that surpasses the friction force that is needed to prevent falling. 
Instead, the use of a rope/wire introduces an external force to compensate for gravity and allow the contact 
forces to better satisfy the friction constraints (i.e., lay more in the middle of the friction cones) when 
walking on highly inclined terrains.

%% Third advantage : increase speed of movment 
Furthermore, with respect to most common climbing robots that uses sticky pads and gaits 
to climb a wall \cite{Kim2008}, \cite{Riskin2009}, the aid of a rope as the recently bio-inspired 
dragline locomotion from spiders in \cite{Wang2014} can dramatically increase the locomotion speed (R2) by a 
winding and releasing mechanism, therefore being a preferable solution in applications that require a 
prompt intervention such as: search and rescue missions. 
% e.g. efficiency when rappelling down (i.e. exploiting the effect of gravity to unwind the rope.)  

%Last advantage of rope 
Finally, the amount of payload that can be carried by sticky/adhesive based climbing 
robots is very limited due to the small tangential component of the adhesive and leg actuation. 
While, a feature of rope being inextensible (limited by the maximum force deliverable by 
the winding mechanism, e.g. an hoist which can be extended using a gearbox) will allow the robot 
to carry a much higher amount of payload (R3). 

%Difficulties due to the dynamics when using a rope
After all, having the robot attached to a wire poses some challenges: one, is the 
under-actuation where the robot is not able to control the position of its center of mass 
while not in contact and the other, is that a rope represents a unilateral constraint 
(can only pull and not push), which further complicates the already hybrid dynamics and 
the low control authority of this class of robots. 

%Providing a solution for second problem: locomotion technique, we can reach the target with jumps 
Secondly, rather than slowly taking steps as walking technique, the robot can take one or 
multiple horizontal jumps to reach the desired location. For each jump, the winding/releasing 
mechanism can be exploited to behave like a variable length pendulum (see Fig. \ref{fig:sp}). 
The key feature of jumping with rope relies in the ability to address terrains of up to vertical 
inclinations in short time. Salto-1P is vertical jumping robot that can showed as an example for 
the jumping locomotion in \cite{Haldane2017}. 

%Difficulties in the jumping technique
However, taking a jump with a rope that can be released, will make the resulting motion of the robot (and 
so the possibility to reach the target) depend on two factors: 1) the initial impulse that the robot 
exerts on the wall at lift-off, 2) the time-law needed for the rope to be released/re-winded.

%Optimization for plannig 
Therefore, a planning strategy for these kind of motions, to be successful, it should take 
into account both these two factors, the under-actuation of the robot and the 
physical constraints, namely: the constrains posed by the rope, the actuator limits and the 
contact interaction (i.e. friction). 

Numerical optimization is an attractive solution for this 
planning problems \cite{Nguyen2019,Ding2020} because it allows to jointly cope with these aspects 
in an optimal fashion while ensuring that the constraints are satisfied in a planning horizon resulting 
in a feasible trajectory. Casting the locomotion planning as an optimization problem allows one to 
represent high level tasks and system dynamics using cost functions and constraints. 

In this framework different goals can also be pursued. for instance, minimize energy consumption or hoist effort. conversely, the jump time can be minimized to achieve the target in a shortest time in case of emergencies and rescue missions. 


\subsection{Proposed Approach and Contribution}
In this work, we present a novel robotic platform called CLIO that is able to 
reach desired targets on a vertical wall. We propose planning approaches 
based on numerical optimization to solve the jumping problem employing a simplified model of 
the dynamics.
To summarize, the contributions of the paper are:
\begin{itemize}
	\item a novel design of a jumping robot platform CLIO 
	\item a planning approach, based on optimal control, to generate a desired jump motion, 
	overcoming an obstacle comparing pros and cons of each approach.
	\item simulations and experiments to validate  the effectiveness of the proposed approach 
	in a simulation with a simplified model and in a realistic (gazebo) simulation with the 
	full dynamics of the robot.
\end{itemize}


\subsection{Outline}

The paper is organized as follows: Section II gives an
overview of the robot ....
on III describes the optimization
problem with the LIP model and how it is used to compute
CoM position velocity and contact force references. Simula-
tions and experiments with Aliengo robot are illustrated in
Section IV. Finally, we draw the conclusions in Section V.

%%% Local Variables:
%%% mode: latex
%%% TeX-master: t
%%% End:
