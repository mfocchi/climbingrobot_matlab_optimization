As anticipated in the system description above, one of the most complex aspect of this system is that part of the control, the $\vect{F}_u$ force, has an impulsive nature.
Its effect can be modeled as an isntantaneous reset of the velocity $\dot{theta}$ (i.e., the state variable $x_4$) and $\dot{\phi}$ (i.e., $x_5$).
This amount of the reset can be related to the intensity of the two components of the impulse by using known state of the art results~\cite{dim17}:
\begin{equation}
  \begin{aligned}
    &x_4(t_0^+) = x_4(t_0^-) + \frac{1}{m x_3(t_0^-)} F_{u,n}\\
    &x_5(t_0^+) = x_5(t_0^-) + + \frac{1}{m x_3(0) \sin x_1(0)} F_{u,t} \quad .
  \end{aligned}
\end{equation}
Due to the imprecisions in the actuators and to the lack of knowledge on the terrain characteristics, the velocities after the reset
could differ from the exepcted values hence perturbing the whole trajectory.
The only possibility for fixing the deviation is to rely on the only control variable available during the flight, i.e., the traction force $\mathbf{F}_r$, and use it within
a feedback control law.
In this section, we will first address the controllability problem, i.e., if such a feedback law can be found for an underactuated system like the one
one in question. Second, we will show how such a feedback control law can be designed.

\noindent
{\bf System Controllability. }


%%% Local Variables:
%%% mode: latex
%%% TeX-master: "../icra22climb"
%%% End:
