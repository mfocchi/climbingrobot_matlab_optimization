As anticipated in the system description above, one of the most
complex aspect of this system is that part of the control, the
$\vect{F}_u$ force, has an impulsive nature.  Its effect can be
modeled as an isntantaneous reset of the velocities $\dot{\theta}$ and
$\dot{\phi}$, i.e., $x_4$ and $x_5$, respectively.  More precisely,
this effect is modelled using known state of the art
results~\cite{dim17}, i.e.
\begin{equation}
  \begin{aligned}
    &x_4(t_0^+) = x_4(t_0^-) + \frac{1}{m x_3(t_0^-)} F_{u,n} ,\\
    &x_5(t_0^+) = x_5(t_0^-) + + \frac{1}{m x_3(0) \sin x_1(0)}
    F_{u,t} .
  \end{aligned}
\end{equation}
Due to the imprecisions in the actuators and to the lack of knowledge
on the terrain characteristics, the velocities after the reset could
differ from the exepcted values hence perturbing the whole trajectory.
The only possibility for fixing the deviation is to rely on the only
control variable available during the flight, i.e., the traction force
$\mathbf{F}_r$, and use it within a feedback control law.  In this
section, we will first address the controllability problem, i.e., if
such a feedback law can be found for an underactuated system like the
one one in question. Second, we will show how such a feedback control
law can be designed.

\noindent
{\bf Local Reachability. }

Let us consider the system canonical form~\eqref{eq:CanonicalForm},
and let us consider that the inputs $F_{u,n}$, $F_{u,t}$ and $F_{r}$
are all fully available to control the system. By computing the
involutive distribution $\Delta_n$, if it is nonsingular the system is
locally reachable~\cite{isidori1985nonlinear}. As such, consider the
system state $x$ in~\eqref{eq:State}, we can rewrite the
dynamics~\eqref{eq:CanonicalForm} as
\begin{equation}
  \label{eq:VecFields}
  \dot x = f(x) + g_{u,n}(x) F_{u,n} + g_{u,t}(x) F_{u,t} + g_r(x)
  F_{r} ,
\end{equation}
and, hence the initial distribution whose columns are
\[
  \Delta_0 = \{g_{u,n}, g_{u,t}, g_r\} .
\]
It is then possible to derive
\[
  \Delta_1 = \{\Delta_0, [f,g_{u,n}], [f,g_{u,t}], [f,g_r]\} ,
\]
where $[\cdot,\cdot]$ are the usual Lie bracket
operators~\cite{isidori1985nonlinear}, i.e.
\[
  [f,g] = \frac{\partial g}{\partial x}f - \frac{\partial f}{\partial
    x}g .
\]
It turns then out that $\mbox{span}\Delta_1 = 6$, i.e., the
distribution is nonsingular and, hence, the system is locally
reachable. As a consequence, we can design a very simple
Lyapunov-based controller using
backstepping~\cite{krstic1995nonlinear}. Let us define the following
Lyapunov function candidate on the subsystem of interest
\[
  V_1(\theta, \phi, l) = \frac{1}{2} [(\theta - \overline\theta)^2 +
  (\phi - \overline\phi)^2 + (l - \overline l)^2] ,
\]
where the quantities $\overline \cdot$ are the desired, final values
and whose time derivative is simply given by
\[
  \dot V_1(\theta, \phi, l) = (\theta - \overline\theta) \omega +
  (\phi - \overline\phi) \psi + (l - \overline l) r . 
\]
Hence, assuming the proportional control laws (with
$k_\theta, k_\phi, k_l > 0$)
\begin{equation}
  \label{eq:LyapBack}
  \begin{aligned}
    \omega & = - k_\theta  (\theta - \overline\theta) , \\
    \psi & = - k_\phi  (\phi - \overline\phi) , \\
    r & = - k_l  (l - \overline l) , 
  \end{aligned}
\end{equation}
we have immediately that
$[\overline\theta, \overline\phi, \overline l]^T$ is an asymptotically
stable equilibrium point for the subsystem. When the entire state $x$
is considered, it is possible to consider the quantities
in~\eqref{eq:LyapBack} as desired quantities and define the following
Lyapunov function candidate
\[
  V(x) = V_1(\theta, \phi, l) + \frac{1}{2} [(\omega -
  \overline\omega)^2 + (\psi + \overline\psi)^2 + (r + \overline r)^2] .
\]
The time derivative becomes
\[
  \begin{aligned}
    \dot V(x) = & \dot V_1(\theta, \phi, l) + (\omega -
    \overline\omega)
    (f_4(x) + \frac{F_{u,n}}{ml} - \dot{\overline\omega}) + \\
    & + (\psi - \overline\psi) (f_5(x) + \frac{F_{u,t}}{ml s_\theta} -
    \dot{\overline\psi}) + \\
    & + (r - \overline r) (f_6(x) + \frac{F_{r}}{m} - \dot{\overline
      r}) ,
  \end{aligned}
\]
where $f_i(x)$ is the $i$-th vector of the vector field $f(x)$
in~\eqref{eq:VecFields} and where we have from~\eqref{eq:LyapBack}
\[
  \begin{aligned}
    \dot{\overline\omega} & = - k_\theta  \omega , \\
    \dot{\overline\psi} & = - k_\phi \psi , \\
    \dot{\overline r} & = - k_l r . 
  \end{aligned}
\]
It is therefore immediate to notice that the Lyapunov candidate time
derivative $\dot V(x)$ is negative definite with the following control
laws (with $k_\omega, k_\psi, k_r > 0$)
\[
  \begin{aligned}
    F_{u,n} & = -ml (f_4(x) + k_\omega (\omega - \overline\omega) + k_\theta  \omega) , \\
    F_{u,t} & = -ml s_\theta (f_5(x) + k_\psi (\psi - \overline\psi) + k_\psi  \psi) , \\
    F_{r} & = -m (f_6(x) + k_r (r - \overline r) + k_l r) , \\
  \end{aligned}
\]
thus proving the asymptotic stability of the equilibrium
$[\overline\theta, \overline\phi, \overline l, 0, 0, 0]^T$ for the
system~\eqref{eq:CanonicalForm}.

In a more realistic case, we can assume that $F_{u,n} = F_{u,t} = 0$
right after the application of the impulsive inputs, thus we will
study how a stabilising Lyapunov-based control law can be designed
using only the input $F_r$. To this end, we first prove that the
system remains locally reachable. To this end, we define the reduced
version of~\eqref{eq:VecFields} as
\[
  \dot x = f(x) + g_r(x) F_{r} ,
\]
where the effect of the impulsive forces is supposed to be stored in
the state $x$ right after their applications. Hence, the initial
distribution is
\[
  \Delta_0 = \{g_r\} .
\]
It is then possible to derive in a recursive way
\[
  \Delta_{n+1} = \{\Delta_n, [f,\Delta_n]\} .
\]
It turns then out that $\mbox{span}\Delta_5 = 6$, i.e., the
distribution is nonsingular and, hence, the system is locally
reachable.

In this case, we assume that a reference trajectory for the angles
$\theta$ and $\phi$ is given, for example given by the first two
of~\eqref{eq:LyapBack}, for which both the first and second derivative
can be computed. We can then consider the following Lyapunov candidate
\[
  V(x) = \frac{1}{2} [(\theta - \overline\theta)^2 + (\phi -
  \overline\phi)^2 + (l - \overline l)^2 + (\omega -
  \overline\omega)^2 + (\psi + \overline\psi)^2 + r^2] ,
\]
whose time derivative is given by
\[
  \dot V(x) = (\theta - \overline\theta) \omega + (\phi -
  \overline\phi) \psi + (l - \overline l) r + h_1(\omega) + h_2(\psi)
  + r (f_6(x) + \frac{F_{r}}{m}),
\]
where
\[
  \begin{aligned}
    h_1(\omega) = & (\omega - \overline\omega) (f_4(x) -
    \dot{\overline\omega}) , \\
    h_2(\psi) = & (\psi - \overline\psi) (f_5(x) -
    \dot{\overline\psi}) . \\    
  \end{aligned}
\]
We can then build a Lyapunov control law that locally controls the
system dynamics.


%%% Local Variables:
%%% mode: latex
%%% TeX-master: "../icra22climb"
%%% End:
