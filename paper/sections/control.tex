As anticipated in the system description above, one of the most
complex aspects of this system is that part of the control, the
$\vect{F}_u$ force, has an impulsive nature.  Its effect can be
modeled as an instantaneous reset of the velocities $\dot{\theta}$ and
$\dot{\phi}$, i.e., $x_4$ and $x_5$, respectively.  More precisely,
this effect is modelled using known state-of-the-art
results~\cite{dim17}, i.e.
\begin{equation}
  \label{eq:Reset}
  \begin{aligned}
    &x_4(t_0^+) = x_4(t_0^-) + \frac{1}{m x_3(t_0^-)} F_{u,n},\\
    &x_5(t_0^+) = x_5(t_0^-) +  \frac{1}{m x_3(0) \sin x_1(0)} F_{u,t} .
  \end{aligned}
\end{equation}
Due to the imprecisions in the actuators and to the lack of knowledge
on the terrain characteristics, the velocities after the reset could
differ from the expected values hence perturbing the whole trajectory.
The only possibility for fixing the deviation is to rely on the only
control variable available during the flight, i.e., the traction force
$F_r$, and use it within a feedback control law.  In this
section, we first address the controllability problem, i.e., if
such a feedback law can be found for an under-actuated system like ours. 
Second, we show how to design such a feedback control law.

\subsection{Local Reachability}

Let us consider the system canonical form~\eqref{eq:NonLinStateSpace},
with the availability of the input $F_r$, while
$F_{u,n} = F_{u,t} = 0$ assuming resets and impulsive forces as
in~\eqref{eq:Reset}. By recalling the results
in~\cite{isidori1985nonlinear}, we can compute the involutive
distribution $\Delta_n(\vect{x})$ and conclude for local reachability
iff $\Delta_n(\vect{x})$ is nonsingular. As such, consider the system
state $\vect{x}$ and the dynamic in~\eqref{eq:NonLinStateSpace},
rewritten as
\begin{equation}
  \label{eq:VecFields}
  \dot {\vect{x}} = f(\vect{x}) + g_r(\vect{x}) F_{r} ,
\end{equation}
hence the initial distribution is given by the column
\[
  \Delta_0(\vect{x}) = \{g_r(\vect{x})\} .
\]
It is then possible to derive
\[
  \Delta_1(\vect{x}) = \{\Delta_0(\vect{x}),
  [f(\vect{x}),g_r(\vect{x})]\} = \{\Delta_0(\vect{x}), [f(\vect{x}),
  \Delta_0(\vect{x})]\} ,
\]
where $[\cdot,\cdot]$ are the usual Lie bracket
operators~\cite{isidori1985nonlinear}, i.e.
\[
  [f(\vect{x}),g_r(\vect{x})] = \frac{\partial g_r(\vect{x})}{\partial
    \vect{x}}f(\vect{x}) - \frac{\partial f(\vect{x})}{\partial
    \vect{x}}g_r(\vect{x}) .
\]
It is then possible to derive in a recursive manner
\[
  \Delta_{n+1}(\vect{x}) = \{\Delta_n(\vect{x}),
  [f(\vect{x}),\Delta_n(\vect{x})]\} ,
\]
where, with a slight abuse of notation,
$[f(\vect{x}),\Delta_n(\vect{x})]$ is intended as the Lie brackets of
$f(\vect{x})$ with respect to all the vector fields of
$\Delta_n(\vect{x})$. After carrying out all the computations, it turns out that
$\dim\{\mbox{span}\{\Delta_5\}\} = 6$, i.e., the distribution is
nonsingular and, hence, the system is locally reachable.

\subsection{Lyapunov-based control design}

One possible path to design a state-feedback, Lyapunov-based
controller for the system~\eqref{eq:NonLinStateSpace} is to use the
backstepping technique~\cite{krstic1995nonlinear}. Let us define the
following Lyapunov function candidate on the position subsystem
\[
  V_1(x_1, x_2, x_3) = \frac{1}{2} [(x_1 - \overline x_1)^2 +
  (x_2 - \overline x_2)^2 + (x_3 - \overline x_3)^2] ,
\]
where the quantities $\overline \cdot$ are the desired, final values
and whose time derivative is simply given by
\[
  \dot V_1(x_1, x_2, l) = (x_1 - \overline x_1) x_4 +
  (x_2 - \overline x_2) x_5 + (x_3 - \overline x_3) x_6 . 
\]
Hence, assuming the proportional control laws (with
$k_{x_1}, k_{x_2}, k_{x_3} > 0$)
\begin{equation}
  \label{eq:LyapBack}
    x_4 = - k_{x_1}  (x_1 - \overline x_1), 
    x_5 = - k_{x_2}  (x_2 - \overline x_2), 
    x_6 = - k_{x_3} (x_3 - \overline x_3),
\end{equation}
we have immediately that
$[\overline x_1, \overline x_2, \overline x_3]^T$ is an asymptotically
stable equilibrium point for the subsystem. When the entire state
$\vect{x}$ is considered, the quantities in~\eqref{eq:LyapBack} are
treated as desired and then define the following Lyapunov function
candidate
\[
  V(\vect{x}) = V_1(x_1, x_2, x_3) + \frac{1}{2} [(x_4 - \overline
  x_4)^2 + (x_5 + \overline x_5)^2 + (x_6 + \overline x_6)^2] .
\]
The time derivative becomes
\[
  \begin{aligned}
    \dot V(\vect{x}) = &(x_4 - \overline x_4) (f_4(\vect{x}) -
    \dot{\overline x}_4) + (x_5 - \overline x_5)
    (f_5(\vect{x}) - \dot{\overline x}_5) + \\
    & + (x_6 - \overline x_6) \left (f_6(\vect{x}) + \frac{F_{r}}{m} -
      \dot{\overline x}_6 \right ) + \dot V_1(x_1, x_2, x_3),
  \end{aligned}
\]
where $f_i(\vect{x})$ is the $i$-th vector of the vector field
$f(\vect{x})$ in~\eqref{eq:VecFields} and where we have
from~\eqref{eq:LyapBack}
\[
  \begin{aligned}
    \dot{\overline x_4} & = - k_{x_1}  x_4 , \\
    \dot{\overline x_5} & = - k_{x_2} x_5 , \\
    \dot{\overline x_6} & = - k_{x_3} r . 
  \end{aligned}
\]
It is therefore immediate to notice that the Lyapunov candidate time
derivative $\dot V(x)$ is negative definite with the following control
laws (with $k_{x_4}, k_{x_5}, k_{x_6} > 0$)
\[
  \begin{aligned}
    F_{u,n} & = -ml (f_4(x) + k_{x_4} (x_4 - \overline x_4) + k_{x_1}  x_4) , \\
    F_{u,t} & = -ml s_{x_1} (f_5(x) + k_{x_5} (x_5 - \overline x_5) + k_{x_5}  x_5) , \\
    F_{r} & = -m (f_6(x) + k_{x_6} (r - \overline x_6) + k_{x_3} r) , \\
  \end{aligned}
\]
thus proving the asymptotic stability of the equilibrium
$[\overline x_1, \overline x_2, \overline x_3, 0, 0, 0]^T$ for the
system~\eqref{eq:CanonicalForm}.

In a more realistic case, we can assume that $F_{u,n} = F_{u,t} = 0$
right after the application of the impulsive inputs, thus we will
study how a stabilising Lyapunov-based control law can be designed
using only the input $F_{x_6}$. To this end, we first prove that the
system remains locally reachable. To this end, we define the reduced
version of~\eqref{eq:VecFields} as
\[
  \dot x = f(x) + g_{x_6}(x) F_{r} ,
\]
where the effect of the impulsive forces is supposed to be stored in
the state $x$ right after their applications. Hence, the initial
distribution is
\[
  \Delta_0 = \{g_{x_6}\} .
\]
It is then possible to derive in a recursive way
\[
  \Delta_{n+1} = \{\Delta_n, [f,\Delta_n]\} .
\]
It turns then out that $\mbox{span}\Delta_5 = 6$, i.e., the
distribution is nonsingular and, hence, the system is locally
reachable.

In this case, we assume that a reference trajectory for the angles
$x_1$ and $x_2$ is given, for example given by the first two
of~\eqref{eq:LyapBack}, for which both the first and second derivative
can be computed. We can then consider the following Lyapunov candidate
\[
  V(x) = \frac{1}{2} [(x_1 - \overline x_1)^2 + (x_2 -
  \overline x_2)^2 + (l - \overline x_3)^2 + (x_4 -
  \overline x_4)^2 + (x_5 + \overline x_5)^2 + r^2] ,
\]
whose time derivative is given by
\[
  \dot V(x) = (x_1 - \overline x_1) x_4 + (x_2 -
  \overline x_2) x_5 + (l - \overline x_3) r + h_1(x_4) + h_2(x_5)
  + r (f_6(x) + \frac{F_{r}}{m}),
\]
where
\[
  \begin{aligned}
    h_1(x_4) = & (x_4 - \overline x_4) (f_4(x) -
    \dot{\overline x_4}) , \\
    h_2(x_5) = & (x_5 - \overline x_5) (f_5(x) -
    \dot{\overline x_5}) . \\    
  \end{aligned}
\]
We can then build a Lyapunov control law that locally controls the
system dynamics.


%%% Local Variables:
%%% mode: latex
%%% TeX-master: "../icra22climb"
%%% End:
