
\subsection{Simplified Model}
We can define a simplified mathematical model by making the following assumptions: 1. The mass is entirely concentrated in the body attached to the wire, 2. We can unwind and rewind the wire but it is rigid and remains completely elongated (i.e., it cannot make bends), 3. The rewinder can pull the wire when winding it but cannot push it (i.e., it can only act as a brake during the unwinding phase).
A geometric sketch of the system is shown in Figure \ref{fig:sp}. where the polar angle is $\theta$ [rad], the Azimuth angle $\phi$ [rad], and $l$ [m] is the length of the rigid rope. 

\begin{figure}
\includegraphics[width=\columnwidth]{figs/spherical_pendulum}
\caption{Logical scheme for the simplified model}
\label{fig:sp}
\end{figure}
Our input variables to the system are: 1. the rewinder pull action or
braking action $\mathbf{F}_r$ oriented along the wire, 2. an impulsive
push force $\mathbf{F}_u$ that the robot can generate when it is
attached to the mountain wall. 

Let $e$ be a frame attached to the mass
and with axis oriented along the $x$ axis, and let $w$ be the world
frame attached to the suspension point of the pendulum.  If we
consider the homogeneous transformation linking $e$ to $w$, we can
write:
	\[
	T_e^w = \begin{bmatrix} R_z(\phi) &\begin{matrix}0\\0\\0 \end{matrix}  \\ \begin{matrix} 0 &0 &0 \end{matrix} & 1\end{bmatrix}  \begin{bmatrix} R_y(\pi/2 - \theta) &\begin{matrix}0\\0\\0 \end{matrix}  \\ \begin{matrix} 0 &0 &0 \end{matrix} & 1\end{bmatrix}  \begin{bmatrix} I_{3,3} &\begin{matrix}0\\0\\0 \end{matrix}  \\ \begin{matrix} L &0 &0 \end{matrix} & 1\end{bmatrix}  
	\]
where $R_z(\phi)$ is the rotation matrix of $\phi$ around $z$ , and $R_y(\alpha)$ is the rotation matrix of $\alpha$ around $y$. 
Simplifying $T_e^w$, gives :
	\[
	T_e^w = \begin{bmatrix} c_\phi s_\theta & -s_\phi & c_\phi c_\theta & l c_\phi s_\theta\\
		s_\phi s_\theta & c_\phi & c_\theta s_\phi & l s_\phi s_\theta\\
		-c_\theta & 0 & s_\theta &-l c_\theta\\
		0 & 0 &0 &1\end{bmatrix} 
	\]		
where $c_x$ is a shorthand for $\cos x$ , and $s_x$ is a shorthand for $\sin x$.
The position $\mathbf{p}$ of the mass can be extracted from the transformation matrix $T_e^w$ as : 
	\begin{align}  
		\mathbf{p} = \begin{bmatrix} x\\ y\\ z \end{bmatrix} =
		\begin{bmatrix}   
            l s_\theta c_\phi\\
			l s_\theta s_\phi\\
			-l c_\theta \end{bmatrix} \label{eq1}
    \end{align}
%From equation \ref{eq1}, we can compute the velocities and the kinetic energy:   
From equation \eqref{eq1}, the velocities along Cartesian axes are    
\begin{align*}
	\dot{x} &= 
	l c_\theta c_\phi \dot{\theta} - l s_\theta s_\phi \dot{\phi} + s_\theta c_\phi \dot{l} \\
	\dot{y} &= l c_\theta s_\phi \dot{\theta} + l s_\theta c_\phi \dot{\phi} + s_\theta s_\phi \dot{l} \\
	\dot{z} &= l s_\theta  \dot{\theta} - c_\theta \dot{l} 
\end{align*}
Next, the velocity squared of the mass 
\begin{align*}
	v^2 &= \dot{x}^2 + \dot{y}^2 + \dot{z}^2 =  l^2 \dot{\theta}^2 + l^2 s_\theta^2 \dot{\phi}^2 + \dot{l}^2 		
%	&=  l^2 \dot{\theta}^2 + l^2 s_\theta^2 \dot{\phi}^2 + \dot{l}^2\\
\end{align*}
Thus, the kinetic energy $T$ and potential energy $V$  
\begin{align*}
	T &=\frac{ m }{2} v^2 =  \frac{m}{2} l^2 \left(\dot{\theta}^2 + s_\theta^2 \dot{\phi}^2 \right) + \frac{m }{2} \dot{l}^2  \\
    V &= mgz = -mgl c_\theta   \label{eq3}
\end{align*}
which leading to the Lagrangian described as
\begin{equation}
	L = T - V = \frac{m}{2} l^2 \left(\dot{\theta}^2 + s_\theta^2 \dot{\phi}^2\right) + \frac{m}{2}\dot{l}^2 + mgl c_\theta
\end{equation}
The dynamics of the system will be obtained using the Euler-Lagrange equation:
\[
\frac{d}{dt}\frac{\partial L}{\partial \dot{q}_i} - \frac{\partial L}{\partial q_i} = Q_i^p ,
\]
where $Q_i^p = (\mathbf{F}_r + \mathbf{F}_u) \cdot \frac{\partial \mathbf{p}}{\partial q_i}$ is the generalized force, 	
	$\mathbf{F}_r$ is oriented along the rope, and $\mathbf{F}_u$ is generated so that it does not have any component along the rope.
	\begin{equation}
	\begin{aligned}
		\mathbf{F}_r  &= F_r \begin{bmatrix}
			c_\phi s_\theta\\
			s_\phi s_\theta\\
			-c_\theta
		\end{bmatrix}^T = F_r \mathbf{f}_r \\	
		\mathbf{F}_u &= F_{u,t}\begin{bmatrix}
			-s_\phi\\
			c_\phi\\
			0
		\end{bmatrix}^T + F_{u,n} \begin{bmatrix}
			c_\phi c_\theta \\
			s_\phi c_\theta\\
			s_\theta
		\end{bmatrix}^T &= F_{u,t} \mathbf{f}_{u,t} + F_{u,n} \mathbf{f}_{u,n}    	 
	\end{aligned}
	\label{eq:forces}
	\end{equation}
	Let the generalized coordinates $q_i$ be chosen as $q_1 = \theta$, $q_2 = \phi$ and $q_3 = l$ . The corresponding velocities are 
	\begin{align*}
		\frac{\partial \mathbf{p}}{\partial q_1} = \frac{\partial \mathbf{p}}{\partial \theta} &= \begin{bmatrix}
			l c_\phi c_\theta\\
			l s_\phi c_\theta\\
			l s_\theta
		\end{bmatrix},  \quad \text{and} \quad 
		\frac{\partial \mathbf{p}}{\partial q_2} = \frac{\partial \mathbf{p}}{\partial \phi} &= \begin{bmatrix}
			- l s_\phi s_\theta\\
			l c_\phi s_\theta\\
			0
		\end{bmatrix}, \\
		\frac{\partial \mathbf{p}}{\partial q_3} = \frac{\partial \mathbf{p}}{\partial l} &= \begin{bmatrix}
			c_\phi s_\theta\\
			s_\phi s_\theta\\
			-c_\theta
		\end{bmatrix} .
	\end{align*}
	Hence,
\begin{align*}
	Q_1^p &= \left(\mathbf{F}_r + \mathbf{F}_u\right)   \frac{\partial \mathbf{p}}{\partial q_1} 
	%% 		&= F_r \mathbf{f}_r \cdot  \frac{\partial \mathbf{p}}{\partial \theta} + \\
	%% 		&+ F_{u,t} \mathbf{f}_{u,t} \cdot  \frac{\partial \mathbf{p}}{\partial \theta} + \\
	%% 		&+ F_{u,n} \mathbf{f}_{u,n} \cdot  \frac{\partial \mathbf{p}}{\partial \theta}  =  \\
	%% 		&= F_{u,n} \mathbf{f}_{u,n} \cdot  \frac{\partial \mathbf{p}}{\partial \theta} = \\
	= F_{u,n} l , \\
	% 	\end{align*}
% 	\begin{align*}
	Q_2^p &= \left(\mathbf{F}_r + \mathbf{F}_u\right)   \frac{\partial \mathbf{p}}{\partial q_2}
	%% 		&= F_r \mathbf{f}_r \cdot  \frac{\partial \mathbf{p}}{\partial \phi} + \\
	%% 		&+ F_{u,t} \mathbf{f}_{u,t} \cdot  \frac{\partial \mathbf{p}}{\partial \phi} + \\
	%% 		&+ F_{u,n} \mathbf{f}_{u,n} \cdot  \frac{\partial \mathbf{p}}{\partial \phi}   = \\
	%% 		&= F_{u,t} \mathbf{f}_{u,t} \cdot  \frac{\partial \mathbf{p}}{\partial \phi} =\\
	= F_{u,t} l s_\theta,\\
	%% 	\end{align*}
%% 	\begin{align*}
	Q_3^p &= \left(\mathbf{F}_r + \mathbf{F}_u\right)   \frac{\partial \mathbf{p}}{\partial q_3}
	%% 		&= F_r \mathbf{f}_r \cdot  \frac{\partial \mathbf{p}}{\partial l} + \\
	%% 		&+ F_{u,t} \mathbf{f}_{u,t} \cdot  \frac{\partial \mathbf{p}}{\partial l} + \\
	%% 		&+ F_{u,n} \mathbf{f}_{u,n} \cdot  \frac{\partial \mathbf{p}}{\partial l}   = \\
	%% 		&= F_r \mathbf{f}_r \cdot  \frac{\partial \mathbf{p}}{\partial l} =\\
	= F_{r} .
\end{align*}
With the defined generalized coordinates, The Euler-Lagrange equations gives us : 	
%% 	The first Euler-Lagrange equation yields
 	\begin{align*}
%		&\frac{d}{dt}\left(\frac{\partial L}{\partial \dot{\theta}} \right) - \frac{\partial L}{\partial \theta} = Q_1^p  \rightarrow  \\
%%		& \frac{d}{dt}\left(m l^2 \dot{\theta}\right) - (m l^2 s_\theta c_\theta \dot{\phi}^2 - m g l s_\theta) = F_{u,n}  l \\ 
 		  &m l^2 \ddot{\theta} + 2 m l \dot{\theta} \dot{l} - m l^2 s_\theta c_\theta \dot{\phi}^2 + m g l s_\theta = F_{u,n} . l\\
%% 		& \ddot{\theta} + \frac{2}{l} \dot{\theta} \dot{l} - c_\theta s_\theta \dot{\phi}^2 + \frac{g}{l}  s_\theta =  \frac{F_{u,n}  }{ml}    
%% 	\end{align*}
%% 	The second Euler-Lagrange equation yields
%% 	\begin{align*}
%% 		&\frac{d}{dt}\left(\frac{\partial L}{\partial \dot{\phi}} \right) - \frac{\partial L}{\partial \phi} = Q_2^p    \\
%% 		&\frac{d}{dt}\left(m l^2 s_\theta^2 \dot{\phi}\right) = F_{u,t}  l  s_\theta\\
 		 &\ddot{\phi} ml^2 s^2_\theta + 2 m l \dot{l} s_\theta^2 \dot{\phi} + 2 m l^2 s_\theta c_\theta \dot{\theta} \dot{\phi}  = F_{u,t}  l  s_\theta\\
%%		&\ddot{\phi} + 2 \frac{c_\theta}{s_\theta} \dot{\theta} \dot{\phi} + \frac{2}{l} \dot{\phi} \dot{l} = \frac{F_{u,t}}{ml s_\theta}
%% 	\end{align*}
%% 	The third Euler-Lagrange equation is as follows:
%% 	\begin{align*}
%% 		&\frac{d}{dt}\left(\frac{\partial L}{\partial \dot{l}} \right) - \frac{\partial L}{\partial l} = Q_3^p    \\
%% 		&\frac{d}{dt}\left(m\dot{l}\right) - (m l \dot{\theta}^2 + m l s^2_\theta \dot{\phi}^2 + m g c_\theta) = F_r ,
 		&m \ddot{l} - m l \dot{\theta}^2 - m l s^2_\theta \dot{\phi}^2 - m g c_\theta = F_r 
%% 		&\ddot{l} - l \dot{\theta}^2 - l s^2_\theta \dot{\phi}^2 - g c_\theta = \frac{F_r}{m} 
 	\end{align*}
           %%
        Overall, the non-linear dynamic equations of the system 
%        Overall, the nonlinear equations of the system are:
        \begin{equation}
        \begin{aligned}
		&\ddot{\theta} + \frac{2}{l} \dot{\theta} \dot{l} - c_\theta s_\theta  \dot{\phi}^2 + \frac{g}{l}  s_\theta =  \frac{1}{ml}F_{u,n} \\
		&\ddot{\phi} + 2 \frac{c_\theta}{s_\theta} \dot{\theta} \dot{\phi} + \frac{2}{l} \dot{\phi} \dot{l} = \frac{1}{ml s_\theta} F_{u,t} \\
		&\ddot{l} - l \dot{\theta}^2 - l s^2_\theta \dot{\phi}^2 - g c_\theta = \frac{1}{m} F_r .
	\end{aligned}
        \label{eq:nonlinearDyn}
        \end{equation}
        Clearly, in this derivation of the model, we have heavily
        relied on the point-mass nature of the body.  A possible issue
        could be the rotational dynamics of the body around its axis,
        which could waste energy generating unnecessary motions and
        impede the landing phase. This issue will be part of our
        future work. However, as discussed next, under reasonable
        assumptions (e.g., thrusting force oriented in the direction
        of the center of mass), the results based on the simplified
        model can be applied to a realistic system with a good approximation.

\subsection{Robot motion: problems and solution overview}
The problem of moving CLIO between two given configurations is not an
easy task.  First of all, even the simplified dynamics in
Equation~\eqref{eq:nonlinearDyn} is highly nonlinear.  Second, 
the configurations the robot moves between can be in general two 
distant points of the workspace. Therefore, it is not possible to
use the linearized dynamics.  Third, one of our
actuators, the thrusting force $\mathbf{F}_u$, has an impulsive nature
and operates at discrete time instants. Therefore, it is not possible
to use this part of the actuation in any feedback control scheme.
Finally, the way the tangential component $F_{u,t}$ is generated is by
using friction. Therefore, the two components $F_{u,n}$ and $F_{u,t}$ are linked
by a nonlinear constraints expressing the friction cone. Because the friction cone
depends on the specific point or area where the robot is pointing its leg,
this mechanism is not totally reliable (i.e., there can be significant deviation
between the generated and the exacted values of $F_{u,t}$).

In view of this complexity, we propose an approach based on two steps:
1. first a motion strategy is planned prior to starting the motion,
2. after the take of phase a motion controller operates on
$\mathbf{F}_r$ in a feedback control scheme to fix small deviation and
to secure that the robot lands close enough to the expected position.

