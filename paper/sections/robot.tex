
\subsection{Simplified Model}
We can define a simplified mathematical model by making the following assumptions: 1. The mass is entirely concentrated in the body attached to the wire, 2.  The wire can be unwounded and rewounded but it is rigid and remains completely elongated (i.e., it cannot make bends), 3. The rewinder can pull the wire when winding it but cannot push it (i.e., it can only act as a brake during the unwinding phase).
A geometric sketch of the system is reported in Figure \ref{fig:sp}. where $\theta$ is the polar angle [rad] and $\phi$ is the Azimuth angle [rad] .  

\begin{figure}
\includegraphics[width=\columnwidth]{figs/spherical_pendulum}
\caption{Logical scheme for the simplified model}
\label{fig:sp}
\end{figure}
Our input variable to the system are: 1. the rewinder pull action or
braking action $\mathbf{F}_r$ oriented along the wire, 2. an impulsive
push force $\mathbf{F}_u$ that the robot can generate when it is
attached to the mountain wall Let $e$ be a frame attached to the mass
and with axis oriented along the $x$ axis and let $w$ be the world
frame attached to the suspension point of the pendulum.  If we
consider the homogeneous transformation linking $e$ to $w$, we can
write:
	\[
	T_e^w = \begin{bmatrix} R_z(\phi) &\begin{matrix}0\\0\\0 \end{matrix}  \\ \begin{matrix} 0 &0 &0 \end{matrix} & 1\end{bmatrix}  \begin{bmatrix} R_y(\pi/2 - \theta) &\begin{matrix}0\\0\\0 \end{matrix}  \\ \begin{matrix} 0 &0 &0 \end{matrix} & 1\end{bmatrix}  \begin{bmatrix} I_{3,3} &\begin{matrix}0\\0\\0 \end{matrix}  \\ \begin{matrix} L &0 &0 \end{matrix} & 1\end{bmatrix}  
	\]
	where $R_z(\phi)$ is the rotation matrix of $\phi$ around $z$ and $R_y(\alpha)$ is the rotation matrix of $\alpha$ around $y$.
	Overall, we have:
	\[
	T_e^w = \begin{bmatrix} c_\phi s_\theta & -s_\phi & c_\phi c_\theta & l c_\phi s_\theta\\
		s_\phi s_\theta & c_\phi & c_\theta s_\phi & l s_\phi s_\theta\\
		-c_\theta & 0 & s_\theta &-l c_\theta\\
		0 & 0 &0 &1\end{bmatrix}
	\]
	where $c_x$ is a shorthand for $\cos x$ and $s_x$ is a shorthand for $\sin x$.
	From this equation, we have that the position $\mathbf{p}$ of the mass is given by:
	\begin{align} \label{eq1}
		\mathbf{p} = \begin{bmatrix} x\\ y\\ z \end{bmatrix} =
		\begin{bmatrix} l s_\theta c_\phi\\
			l s_\theta s_\phi\\
			-l c_\theta \end{bmatrix}
	\end{align}
%% 	From equation \ref{eq1}, the velocities along the axes are:  
%% 	\begin{align*}
%% 		\dot{x} &= 
%% 		l c_\theta c_\phi \dot{\theta} - l s_\theta s_\phi \dot{\phi} + s_\theta c_\phi \dot{l} \\
%% 		\dot{y} &= l c_\theta s_\phi \dot{\theta} + l s_\theta c_\phi \dot{\phi} + s_\theta s_\phi \dot{l} \\
%% 		\dot{z} &= l s_\theta  \dot{\theta} - c_\theta \dot{l} 
%% 	\end{align*}
%% Next, the velocity of the mass is :
%% 	\begin{align*}
%% 		v^2 &= \dot{x}^2 + \dot{y}^2 + \dot{z}^2 \\		
%% 		&=  l^2 c_\theta^2 c_\phi^2 \dot{\theta}^2 + l^2 s_\theta^2 s_\phi^2 \dot{\phi}^2 + s_\theta^2 c_\phi^2 \dot{l}^2 - 2l^2 c_\theta s_\theta c_\phi s_\phi   \dot{\theta} \dot{\phi} - 2l s_\theta^2 s_\phi c_\phi \dot{l} \dot{\phi} \\
%% 		& + 2l c_\theta s_\theta c_\phi^2  \dot{\theta} \dot{l} + l^2 c_\theta^2 s_\phi^2 \dot{\theta}^2 + l^2 s_\theta^2 c_\phi^2 \dot{\phi}^2 + s_\theta^2 s_\phi^2 \dot{l}^2 + 2l^2 c_\theta s_\theta s_\phi c_\phi \dot{\theta} \dot{\phi} \\
%% 		&+ 2l s_\theta^2 c_\phi s_\phi \dot{l} \dot{\phi} + 2l c_\theta s_\theta s_\phi^2  \dot{\theta} \dot{l} + l^2 s_\theta^2  \dot{\theta}^2 + c_\theta^2 \dot{l}^2 - 2l s_\theta c_\theta   \dot{\theta} \dot{l} \\ \\
%% 		&= l^2 c_\theta^2 \left(c_\phi^2 + s_\phi^2\right) \dot{\theta}^2 + l^2 s_\theta^2  \left( s_\phi^2 + c_\phi^2 \right) \dot{\phi}^2 + l^2 s_\theta^2  \dot{\theta}^2 + \dot{l}^2 s_\theta^2 \left(c_\phi^2 + s_\phi^2\right) \\
%% 		&+ 2l c_\theta s_\theta \left(c_\phi^2 + s_\phi^2\right) \dot{\theta} \dot{l} + c_\theta^2 \dot{l}^2 - 2l s_\theta c_\theta \dot{\theta} \dot{l} \\  \\
%% 		&=  l^2 \dot{\theta}^2 + l^2 s_\theta^2 \dot{\phi}^2 + \dot{l}^2\\
%% 	\end{align*}
	
%% 	Thus, the kinetic energy $T$ of the system :
%% 	\begin{align*}
%% 		T &=\frac{ m }{2} v^2 \\
%% 		&= \frac{m}{2} \left(\dot{x}^2 + \dot{y}^2 + \dot{z}^2\right) \\
%% 		&=  \frac{m}{2} l^2 \left(\dot{\theta}^2 + s_\theta^2 \dot{\phi}^2 \right) + \frac{m }{2} \dot{l}^2 \\
%% 	\end{align*}
%% And the potential energy $V$ :
%% 		\begin{align*}
%% 		V &= mgz\\
%% 		&= -mgl c_\theta
%% 	\end{align*}
%%  The Lagrangian function is given by:
%% 	\begin{equation}
%% 		L = T - V = \frac{m}{2} l^2 \left(\dot{\theta}^2 + s_\theta^2 \dot{\phi}^2\right) + \frac{m}{2}\dot{l}^2 + mgl c_\theta
%% 	\end{equation}
%% 	The generalized coordinates are in this case given by $q_1 = \theta, q_2 = \phi,$ and $q_3 = l$ .
%% 	If we apply the d'Alembert principle, the Lagrangian Equation can be written as:
%% 	\[
%% 	\frac{d}{dt}\frac{\partial L}{\partial \dot{q}_i} - \frac{\partial L}{\partial q_i} = Q_i^p ,
%% 	\]
%% 	where $Q_i^p$ is the generalized force $Q_i^p = (\mathbf{F}_r + \mathbf{F}_u) \cdot \frac{\partial \mathbf{p}}{\partial q_i}$.
%% 	Observe that  $\mathbf{F}_r$ oriented along the rope and $\mathbf{F}_u$ does not have any component along the rope (otherwise
%% 	it will forms bends):
%% 	\begin{align*}
%% 		\mathbf{F}_r  &= F_r \begin{bmatrix}
%% 			c_\phi s_\theta\\
%% 			s_\phi s_\theta\\
%% 			-c_\theta
%% 		\end{bmatrix}^T = F_r \mathbf{f}_r \\	
%% 		\mathbf{F}_u &= F_{u,t}\begin{bmatrix}
%% 			-s_\phi\\
%% 			c_\phi\\
%% 			0
%% 		\end{bmatrix}^T + F_{u,n} \begin{bmatrix}
%% 			c_\phi c_\theta \\
%% 			s_\phi c_\theta\\
%% 			s_\theta
%% 		\end{bmatrix}^T &= F_{u,t} \mathbf{f}_{u,t} + F_{u,n} \mathbf{f}_{u,n}    \\ 	 
%% 	\end{align*}  
%% 	while
%% 	\begin{align*}
%% 		\frac{\partial \mathbf{p}}{\partial q_1} = \frac{\partial \mathbf{p}}{\partial \theta} &= \begin{bmatrix}
%% 			l c_\phi c_\theta\\
%% 			l s_\phi c_\theta\\
%% 			l s_\theta
%% 		\end{bmatrix}\\
%% 		\frac{\partial \mathbf{p}}{\partial q_2} = \frac{\partial \mathbf{p}}{\partial \phi} &= \begin{bmatrix}
%% 			- l s_\phi s_\theta\\
%% 			l c_\phi s_\theta\\
%% 			0
%% 		\end{bmatrix}\\
%% 		\frac{\partial \mathbf{p}}{\partial q_3} = \frac{\partial \mathbf{p}}{\partial l} &= \begin{bmatrix}
%% 			c_\phi s_\theta\\
%% 			s_\phi s_\theta\\
%% 			-c_\theta
%% 		\end{bmatrix}\\
%% 	\end{align*}
%% 	Hence,
%% 	\begin{align*}
%% 		Q_1^p &= \left(\mathbf{F}_r + \mathbf{F}_u\right)   \frac{\partial \mathbf{p}}{\partial q_1}\\
%% 		&= F_r \mathbf{f}_r \cdot  \frac{\partial \mathbf{p}}{\partial \theta} + \\
%% 		&+ F_{u,t} \mathbf{f}_{u,t} \cdot  \frac{\partial \mathbf{p}}{\partial \theta} + \\
%% 		&+ F_{u,n} \mathbf{f}_{u,n} \cdot  \frac{\partial \mathbf{p}}{\partial \theta}  =  \\
%% 		&= F_{u,n} \mathbf{f}_{u,n} \cdot  \frac{\partial \mathbf{p}}{\partial \theta} = \\
%% 		&= F_{u,n} l ,
%% 	\end{align*}
%% 	\begin{align*}
%% 		Q_2^p &= \left(\mathbf{F}_r + \mathbf{F}_u\right)   \frac{\partial \mathbf{p}}{\partial q_2}\\
%% 		&= F_r \mathbf{f}_r \cdot  \frac{\partial \mathbf{p}}{\partial \phi} + \\
%% 		&+ F_{u,t} \mathbf{f}_{u,t} \cdot  \frac{\partial \mathbf{p}}{\partial \phi} + \\
%% 		&+ F_{u,n} \mathbf{f}_{u,n} \cdot  \frac{\partial \mathbf{p}}{\partial \phi}   = \\
%% 		&= F_{u,t} \mathbf{f}_{u,t} \cdot  \frac{\partial \mathbf{p}}{\partial \phi} =\\
%% 		&= F_{u,t} l s_\theta,
%% 	\end{align*}
%% 	\begin{align*}
%% 		Q_3^p &= \left(\mathbf{F}_r + \mathbf{F}_u\right)   \frac{\partial \mathbf{p}}{\partial q_3}\\
%% 		&= F_r \mathbf{f}_r \cdot  \frac{\partial \mathbf{p}}{\partial l} + \\
%% 		&+ F_{u,t} \mathbf{f}_{u,t} \cdot  \frac{\partial \mathbf{p}}{\partial l} + \\
%% 		&+ F_{u,n} \mathbf{f}_{u,n} \cdot  \frac{\partial \mathbf{p}}{\partial l}   = \\
%% 		&= F_r \mathbf{f}_r \cdot  \frac{\partial \mathbf{p}}{\partial l} =\\
%% 		&= F_{r} .
%% 	\end{align*}
	
	
	
%% 	The first Euler-Lagrange equation yields
%% 	\begin{align*}
%% 		&\frac{d}{dt}\left(\frac{\partial L}{\partial \dot{\theta}} \right) - \frac{\partial L}{\partial \theta} = Q_1^p    \\
%% 		&\frac{d}{dt}\left(m l^2 \dot{\theta}\right) - (m l^2 s_\theta c_\theta \dot{\phi}^2 - m g l s_\theta) = F_{u,n} . l \\ 
%% 		&  m l^2 \ddot{\theta} + 2 m l \dot{\theta} \dot{l} - m l^2 s_\theta c_\theta \dot{\phi}^2 + m g l s_\theta = F_{u,n} . l\\
%% 		& \ddot{\theta} + \frac{2}{l} \dot{\theta} \dot{l} - c_\theta s_\theta \dot{\phi}^2 + \frac{g}{l}  s_\theta =  \frac{F_{u,n}  }{ml}    
%% 	\end{align*}
	
%% 	The second Euler-Lagrange equation yields
%% 	\begin{align*}
%% 		&\frac{d}{dt}\left(\frac{\partial L}{\partial \dot{\phi}} \right) - \frac{\partial L}{\partial \phi} = Q_2^p    \\
%% 		&\frac{d}{dt}\left(m l^2 s_\theta^2 \dot{\phi}\right) - 0 = F_{u,t} . l . s_\theta\\
%% 		& \ddot{\phi} ml^2 s^2_\theta + 2 m l \dot{l} s_\theta^2 \dot{\phi} + 2 m l^2 s_\theta c_\theta \dot{\theta} \dot{\phi}  = F_{u,t} . l . s_\theta\\
%% 		&\ddot{\phi} + 2 \frac{c_\theta}{s_\theta} \dot{\theta} \dot{\phi} + \frac{2}{l} \dot{\phi} \dot{l} = \frac{F_{u,t}}{ml s_\theta}
%% 	\end{align*}
	
%% 	The third Euler-Lagrange equation is as follows:
%% 	\begin{align*}
%% 		&\frac{d}{dt}\left(\frac{\partial L}{\partial \dot{l}} \right) - \frac{\partial L}{\partial l} = Q_3^p    \\
%% 		&\frac{d}{dt}\left(m\dot{l}\right) - (m l \dot{\theta}^2 + m l s^2_\theta \dot{\phi}^2 + m g c_\theta) = F_r \\
%% 		&m \ddot{l} - m l \dot{\theta}^2 - m l s^2_\theta \dot{\phi}^2 - m g c_\theta = F_r \\
%% 		&\ddot{l} - l \dot{\theta}^2 - l s^2_\theta \dot{\phi}^2 - g c_\theta = \frac{F_r}{m} 
%% 	\end{align*}
%% 	Overall, the nonlinear equations of the system are:
%% 	\begin{align}
%% 		&\ddot{\theta} + \frac{2}{l} \dot{\theta} \dot{l} - c_\theta s_\theta  \dot{\phi}^2 + \frac{g}{l}  s_\theta =  \frac{1}{ml}F_{u,n} \\
%% 		&\ddot{\phi} + 2 \frac{c_\theta}{s_\theta} \dot{\theta} \dot{\phi} + \frac{2}{l} \dot{\phi} \dot{l} = \frac{1}{ml s_\theta} F_{u,t} \\
%% 		&\ddot{l} - l \dot{\theta}^2 - l s^2_\theta \dot{\phi}^2 - g c_\theta = \frac{1}{m} F_r 
%% 	\end{align}

