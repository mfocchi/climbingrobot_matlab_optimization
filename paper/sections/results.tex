This section presents simulation results that show that the planned impulses and the pattern of retraction force, 
result in of the optimization in Section \ref{sec:motionP} bring the robot to a desired target. 
In Fig. \ref{fig:validation} we validate  the results of the optimization comparing the trajectories for the \gls{com} 
in the case of the  simplified model (Matlab) and of the full 3D model (Gazebo) for a jump to a target at $p_f = \mat{4 & 5 &-8}$ $[m]$.
The discrepancy in the target position is mainly due to drift in the integration in the case of Matlab and to approximations in the model in case if Gazebo. 
However in both cases the norm of the error is always below XX which shows the simplified model is a good approximation for the real system.
The friction constraints are also fulfilled therefore initially the robot has to lift-off 
with an angle limited by the friction cone. However, since the target location is out of the cone (with vertex at the initial position)
the optimizer should "steer" the trajectory and therefore it dictates  an initial retraction of the rope to vary the time constant of the system, next it passivelt realeases the rope
under the action of gravity.

Additionally, we ran the optimization for  several targets on the wall (see Fig. \ref{fig:targets}) and we 
validated the output of the optimization both  with the simplified and the detailed model. 
Each optimoization provided the initial impulses ($Fun$, $Fut$), the pattern of the winding force $F_r$ and the jump duration $T_f$.
We terminate each simulation after the jump time $T_f$ is elapsed and computed the norm of the error from the target. 
The results are the reported in Table \ref{tab:sim_different_targets}. Note that, as expected, the error increases with the distance of the target.

\begin{figure}
	\includegraphics[width=\columnwidth]{figs/targets.png}
	\caption{Simulation. The red line is  the trajectory of the \gls{com} computed by the optimization, 
		the blue and black line is the simulated trajectory with Matlab and Gazebo, respectively.}
	\label{fig:targets}
\end{figure}



\begin{table}[h!]
	\caption{ Matlab simulations results}
	%	\renewcommand{\arraystretch}{1.1}
	\begin{center}
		\begin{tabular}{@{} l l l l l l @{}} \hline\hline
			\textbf{Test $\#$} \quad & \textbf{Target}           & \textbf{$T_f$} &  \textbf{$F_{un}$} & \textbf{$F_{ut}$} & \textbf{$\Vert e \Vert$ }\\ 
					    1    		 & \mat{4.0  & 5.0 &  -8} &      1.       &                      &                 &                 \\
					    2    		 & \mat{4.0  & 5.0 &  -8} &      1.       &                      &                 &                 \\
					    3    		 & \mat{4.0  & 5.0 &  -8} &      1.       &                      &                 &                 \\
					    4    		 & \mat{4.0  & 5.0 &  -8} &      1.       &                      &                 &                 \\
					    5    		 & \mat{4.0  & 5.0 &  -8} &      1.       &                      &                 &                 \\
					     6    		 & \mat{4.0  & 5.0 &  -8} &      1.       &                      &                 &                 \\
			\hline\hline 					    					    					    
		\end{tabular}
		\label{tab:sim_different_targets}
	\end{center}
\end{table}


\begin{figure}
	\includegraphics[width=\columnwidth]{matlab/sim_results.pdf}
	\caption{Simulation. Validation of the optimization results. The red line is the  trajectory of the \gls{com} computed by the optimization. 
		The blue and black lines are the simulated trajectory with Matlab and Gazebo, respectively.}
	\label{fig:validation}
\end{figure}