This section presents simulation results that show that the planned impulses and the pattern of retraction force, 
result in of the optimization in Section \ref{sec:motionP} bring the robot to a desired target. 

We validate  the results of the optimization comparing the trajectories for the \gls{com} 
in the case of the  simplified model (Matlab) and of the full 3D model (Gazebo) for a jump to a target at $p_f = \mat{4 & 5 &-8}^T$ $[m]$.

The physical parameters of the robot and environment and the optimizarion settings are reported in Table \ref{tab:params}:
\begin{table}[h!]
	\centering
	\caption{ Simulations parameters}
		\begin{tabular}{l c c  } \hline\hline
			\textbf{Name} \quad & \textbf{Symbol}           & \textbf{Value}  \\ \hline
					Robot mass            & m  & 5                 \\ 
					Max. impulse  [N]	& $F_{u,max}$				& 1000   \\
					Max. retraction force [N] & $F_{r,max}$			& 200    \\
					Friction coeff.		& $ \mu $ 					& 0.8   \\
					Thrust impulse duration	[s]& 	$T_{th}$  				& 0.025\\
					Discretization steps  & N & 40\\					        
			\hline\hline 					    					    					    
		\end{tabular}
		\label{tab:params}
\end{table}
%
For the Matlab simulation we simply integrate equations \ref{eq:nonlinearDyn} with a ode45 RK4 variable-step integration scheme.
For the Gazebo simulation,  we employ the full model, therefore we need to find a mapping from the template model 
to the more complex one. We can directly  actuate the rope joint with the $F_r$ force pattern, but we have to  
map the contact force $F$ into efforts at the leg joints 
as follows:
%
\begin{equation}
\tau_{leg}= \mat{\tau_{HP} \\ \tau_{HR} \\ f_{K}} =  h_{leg} -J_{leg}^T \underbrace{\left({}_\mathcal{W} R_b \mat{F_{un} \\ F_{ut} \\ 0} \right) 	}_F
\end{equation}
%
Where $J_{leg} \in \Rnum^{3 \times 3}$ is the sub-matrix of the Jacobian $J$ relative to the leg jonts, ${}_\mathcal{W}R_b$ is the rotation matrix 
that represents the orientation of the base link w.r.t. the inertial frame $\mathcal{W}$ and $h_{leg} \in \Rnum ^3$
represents the bias terms (Centripetal, Coriolis, gravity). Note that  we do
no generate any impulse along the rope direction (base link Z direction) in 
order to avoid to accidentally create  any slack on the rope. 
We set a virtual damper to keep the passive joints at the 
in a fixed position to avoid the motion of the base passive joints, since the optimization is performed 
considering the simplified model that neglects the angular dynamics. 

We set the initial configuration to $q_0= \mat{ atan2(r_{leg}, l_0) & 0 &0  & l_0 & 0 &  0 & 0  & -1.57  & 0 & 0}^T$ 
where $l_0$ is the initial rope length and $r_{leg}=0.38$ $m$ is the leg length in the default configuration. 
At $q_0$  the foot is touching the wall in $p_0 = \mat{0.377, 0, 2.97}^T$ that is the starting point for the optimized trajectory.
A state machine coordinates the 3 phases of the jump: leg orientation, thrusting and flying. In the leg orientation phase, 
 the hip roll joint is commanded
to have the leg aligned with the impulse direction 
by setting the reference for a position PD controller to $q_{HR}^d = atan2(F_{ut}, F_{un})$. 
Note that $q_{0, HP} = -1.57$ in order to have the leg lying on the base frame X-Y axes.
During the \textit{thrusting} phase the force $F$ is generated at the contact 
applying $\tau_{leg}$ for a time $T_{th} = 0.025 s$ then it starts the 
flying phase where the optimized force is set to $\tau_R = F_r$ for the whole jump duration $T_f$.

The computation time for the optimization and the integration error at the target $\Vert e_f \Vert$  are linearly  
linked to the number of discretization points $N$. In Table \ref{tab:solve_time} we report 
also the integration error normalized for the jump length $\Vert e_f \Vert / (l_f-l_0)$ 
for  different numbers of discretization points. 

\begin{table}[h!]
\centering
	%\caption{ Solve time}
	\begin{tabular}{c c c c  } \hline\hline
		\textbf{N} \quad & \textbf{Comp. time [s]}           & \textbf{$\Vert e_f \Vert$ [m]} &  \textbf{$\Vert e_f \Vert / (l_f-l_0)$ [m]} \\ \hline
			250    &   3.             &  X & X \\
			500    &   1.5            & X  & X \\
			1000   &   0.5           & X   & X \\
		\hline\hline 					    					    					    
	\end{tabular}
	\label{tab:solve_time}
\end{table}



In Fig. \ref{fig:validation}  we report the results of the validation. 
The discrepancy in the target position is mainly due to drift in the integration in the case of Matlab and to approximations in the model in case if Gazebo. 
However in both cases the norm of the error is always below XX which shows the simplified model is a good approximation for the real system.
The friction constraints are also fulfilled therefore initially the robot has to lift-off 
with an angle limited by the friction cone. However, since the target location is out of the cone (with vertex at the initial position)
the optimizer should "steer" the trajectory and therefore it dictates  an \textit{initial} 
retraction of the rope to vary the time constant of the "pendulum", 
then the rope passively unwinds  under the action of gravity.
An interesting outcome of this analysis is that despite the system is fully controllable (locally) 
the reachable targets, when jumping  from the anchor line, are limited to the area inside the cone, 
therefore to move tangentially \textit{away} from the anchor line  
only small jumps can be performed. Conversely, the jumps have no limit when jumping toward the anchor line, 
because the gravity component (always pointing toward the anchor line) can be exploited. 

\begin{figure}[H]
	\includegraphics[width=\columnwidth]{matlab/validation.pdf}
	\caption{\small Simulation. Validation of the optimization results. The red line is the  \gls{com} trajectory computed by the optimization. 
		The blue and black lines are the simulated trajectory with Matlab and Gazebo, respectively. 
		The bottom plot is the rope retraction force $F_r(t)$.}
	\label{fig:validation}
\end{figure}

Additionally, we ran the optimization for  several targets on the wall (see Fig. \ref{fig:targets}) and we 
validated the output of the optimization both  with the simplified and the detailed model. 
Each optimization provided the initial impulses ($Fun$, $Fut$), the pattern of the winding force $F_r$ and the jump duration $T_f$.
We terminate each simulation after the jump time $T_f$ is elapsed and computed the norm of the error from the target. 
We selected the target points inside the shaded red area that represents the friction boundaries in the tangential direction.

The results are the reported in Table \ref{tab:sim_different_targets}. 
Note that, as expected, the error increases with the distance of the target.

\begin{figure}
	\includegraphics[width=\columnwidth]{matlab/targets.pdf}
	\caption{\small Simulation. The red line is  the trajectory of the \gls{com} computed by the optimization, 
		the blue and black line is the simulated trajectory with Matlab and Gazebo, respectively.}
	\label{fig:targets}
\end{figure}

\begin{table}[h!]
	\centering
	\caption{ Matlab simulations results}
	%	\renewcommand{\arraystretch}{1.1}
	\resizebox{\columnwidth}{!}{
		\begin{tabular}{c c c c c c  c } \hline\hline
			\textbf{Test} \quad & \textbf{$p_f$ [m]}           & \textbf{$T_f$ [s]} &  \textbf{$F_{un}$ [N]} & \textbf{$F_{ut}$ [N]} 
			\footnote{The value of these forces are related to the selected impulse duration $T_{th}$  [s], they can be strongly reduced by taking longer durations (i.e. in accordance to the actuator  response time).} 
			&\textbf{ $E_{k,f}$ [J]}\\ \hline % & \textbf{$\Vert e \Vert$ [m]}
			1   & $\mat{4.0  & 5.0 &  -8}^T $&      2.16      &         780             &      624           &    408    \\
			2     & $\mat{1.0& 1.0& -8}^T$ &        1.57        &         241             &      192           &    6.5               \\
			3  & $\mat{4.0  & 1.0 &  -8}^T $&      1.2         &       746                  &   200              & 11.8               \\
			4   &$ \mat{2.0  & 2.0 &  -6}^T $&      1.54       &        504              &     403         &      22    &                 \\
			5   &$ \mat{1.0  & 0.0 &  -6}^T $&      0.95     &           180           &      -1.65           &    2.40             \\
			\hline\hline 					    					    					    
		\end{tabular}}
		\label{tab:sim_different_targets}
 \end{table}


