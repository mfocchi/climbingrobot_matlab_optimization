The motion plan for the robot is decided solving an optimal control problem \emph{a la} Pondryagin.
Generally speaking, we can set up the problem in the following way:
\begin{equation}
  \begin{aligned}
    &\text{min}_{\vect{u}(t)} l\left(\vect{x}(t),\vect{u}(t)\right)\\
    &\text{subj. to}\\
    &\,\,\dot{\vect{x}} = f(\vect{x},\vect{u})\\
    &\,\,\vect{u} \in \mathcal{U}\\
    &\,\,\vect{x} \in \mathcal{X}\\
    &\,\,\vect{x}(t_0) = \vect{X}_0, \vect{x}(t_f)=\vect{X}_f.
  \end{aligned}
\end{equation}
where $\vect{x}$ are the state variables, $\vect{u}$ are the input variables, $f(\cdot,\cdot)$ is the dynamic equation of the system,
$\mathcal{U}$ is set of admissible input values, $\mathcal{X}$ is the set of admissible states, $t_0$ and $t_f$ are the initial and final
instants ($t_f$ can also be an optimisation variable) and $\vect{x}_0$ and $\vect{x}_f$ are the desired initial and final states.
The state $\vect{x}$ is in this our problem composed by
$\vect{x}^T = \left[\theta,\,\phi,\,l,\,\dot{\theta},\,\dot{\phi},\, \dot{l}\right]^T$, while the command variables are
given by $\vec{u}^T = \left[\vect{F}_u,\,\vect{F}_t\right]^T$. The dynamic equation is derived from~\eqref{eq:nonlinearDyn}:
\begin{equation}
  \begin{aligned}
    \dot{\begin{bmatrix}
      x_1\\x_2 \\x_3\\x_4\\x_5\\x_6\end{bmatrix}} = \begin{bmatrix}
      x_4\\
      x_5\\ x_6 \\  - \frac{2}{x_3} \dot{x_1} \dot{x_3} + c_{x_1} s_{x_1}  \dot{x_2}^2 - \frac{g}{x_3}  s_{x_1} +  \frac{1}{mx_3}F_{u,n}\\
      - 2 \frac{c_{x_1}}{s_{x_1}} \dot{x_1} \dot{x_2} - \frac{2}{x_3} \dot{x_2} \dot{x_3} + \frac{1}{mx_3 s_{x_1}} F_{u,t}\\
      x_3 \dot{x_1}^2 + x_3 s^2{x_1}\dot{x_2}^2+g c_{x_1}+\frac{1}{m}F_r .
      \end{bmatrix}
  \end{aligned}
\end{equation}
The terminal constraints are usually expressed in the cartesian space $[X,\,Y,\,Z,\,\dot{X},\dot{Y},\dot{Z}]^T$, and they can be can be expressed as a function of
the state variables inverting~\eqref{eq1} and~\eqref{eq:vel}. For instance, for the initial conditions, we have:
\begin{equation}
  \begin{aligned}
    &x_1(t_0)= \atandue\left(-Z_0,\sqrt{X_0^2+Y_0^2}\right), \quad x_2(t_0) = \atandue(Y_0,X_0),\\
    &x_3(t_0) = \sqrt{X_0^2+Y_0^2+Z_0^2},\quad \\
    &x_4(t_0) = \frac{1}{\Lambda}\left(c_{x_{1}(t_0)} c_{x_2(t_0)} \dot{X}_0 +c_{x_1(t_0)}s_{x_2(t_0)} \dot{Y}_0+s_{x_1(t_0)}\dot{Z}_0\right)\\
    &x_5(t_0) = \frac{1}{\Lambda s_{x_1(t_0)}}\left(-s_{x_2(t_0)} \dot{X}_0) \right) + \\
    &\quad  +  \frac{1}{\Lambda s_{x_1(t_0)}} \left(c_{x_2(t_0)} - c_{x_2(t_0)} c^2_{x_1(t_0)} + c^2_{x_1(t_0)} s_{x_2(t_0)}  \right) \dot{Y}_0 +\\
    &\quad + \frac{1}{2 \Lambda}\left(-c_{x_1(t_0)}\left(\sqrt{2}\sin\left(2 x_2(t_0)+\pi/4\right)-1 \right) \right) \dot{Z}_0 \\
    &x_6(t_0) = \frac{x_3(0)}{\Lambda}\left(s_{x_{1}(t_0)} c_{x_2(t_0)} \dot{X}_0 + s_{x_1(t_0)}s_{x_2(t_0)} \dot{Y}_0\right)\\
    & \quad + \frac{1}{2 \Lambda} c_{x_1(t_0)}\left(\sqrt{2} \cos\left(2 x_2(t_0)+\pi/4\right)-\frac{1}{2}\right) \dor{Z}_0 \\
    &\Lambda = x_3(t_0) \left(c_{x_2(t_0)} s_{x_2(t_0)} c^2_{x_1(t_0)} - c^2_{x_2(t_0)} c^2_{x_1(t_0)} + 1\right) \quad .
  \end{aligned}
\end{equation}




%%% Local Variables:
%%% mode: latex
%%% TeX-master: "../icra22climb"
%%% End:
