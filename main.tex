\documentclass[letterpaper, 10 pt, conference]{IEEEtran}  % Comment this line out if you need a4paper

%\documentclass[a4paper, 10pt, conference]{ieeeconf}      % Use this line for a4 paper

\IEEEoverridecommandlockouts                              % This command is only needed if 
% you want to use the \thanks command

%\overrideIEEEmargins                                      % Needed to meet printer requirements.

\usepackage{amsmath, amssymb}
\usepackage{amsfonts}

\let\yesnumberold=\yesnumber\relax
\let\yesnumber\relax
\usepackage[math]{easyeqn}
\let\yesnumber\yesnumberold

\usepackage{bm}
\usepackage{multirow}
\usepackage{graphicx}
\setcounter{MaxMatrixCols}{30}
\usepackage{epstopdf}
\usepackage{enumerate}
\usepackage{color}

\usepackage{booktabs, tabularx}
\usepackage{multirow}

\usepackage{epstopdf}
\usepackage{epsfig}
\usepackage{subfigure}

\usepackage{algorithm}
\usepackage{algpseudocode}




\usepackage{hyperref}

% Label definitions

% Vectors
\def\x{{\mathbf x}}

% Sets
\def\R{\mathbb{R}}
\def\N{\mathbb{N}}
\def\Z{\mathbb{Z}}

% Stochastic
\def\Pr#1{\ensuremath{\text{Pr}\left [#1 \right ]}}
\def\E#1{\ensuremath{\text{E}\left \{#1 \right \}}}

% Matrix measures
\newcommand{\tr}[1]{{\mbox{Trace}\left (#1\right) }}

% CBS
\newcommand{\cbs}{\textsf{CBS}}

% Math operations
\newcommand{\ceil}[1]{\left\lceil #1 \right\rceil}
\newcommand{\floor}[1]{\left\lfloor #1 \right\rfloor}

\DeclareMathOperator*{\argmax}{arg\,max}
\DeclareMathOperator{\softmax}{softmax}

% Frames
\newcommand{\frm}[1]{\langle #1\rangle}

% Theorems
\newtheorem{theorem}{Theorem}
\newtheorem{acknowledgement}[theorem]{Acknowledgement}
%\newtheorem{algorithm}[theorem]{Algorithm}
\newtheorem{axiom}[theorem]{Axiom}
\newtheorem{case}[theorem]{Case}
\newtheorem{claim}[theorem]{Claim}
\newtheorem{conclusion}[theorem]{Conclusion}
\newtheorem{condition}[theorem]{Condition}
\newtheorem{conjecture}[theorem]{Conjecture}
\newtheorem{corollary}[theorem]{Corollary}
\newtheorem{criterion}[theorem]{Criterion}
\newtheorem{definition}[theorem]{Definition}
\newtheorem{example}[theorem]{Example}
\newtheorem{exercise}[theorem]{Exercise}
\newtheorem{lemma}[theorem]{Lemma}
\newtheorem{notation}[theorem]{Notation}
\newtheorem{problem}[theorem]{Problem}
\newtheorem{proposition}{Proposition}
\newtheorem{remark}{Remark}
\newtheorem{solution}[theorem]{Solution}
\newtheorem{summary}[theorem]{Summary}
\newtheorem{assumption}[theorem]{Assumption}
\newtheorem{property}[theorem]{Property}

% Comments
\newcommand{\lui}[1]{\color{green}{{\bf #1}}\color{black}}
\newcommand{\dan}[1]{\color{red}{{\bf #1}}\color{black}}
\newcommand{\ale}[1]{\textcolor{blue}{\bf #1}}
\newcommand{\pla}[1]{\textcolor{cyan}{\bf #1}}
\newcommand{\rev}[1]{\color{blue}{{#1}}\color{black}}

% Cartesian product
\global\long\def\cart{\operatorname*{\diagup\hspace{-9pt}\diagdown}}

% Text rotation
\newcommand\RotText[1]{\rotatebox{90}{\parbox{3.5cm}{\centering#1}}}

\title{\LARGE \bf Climbing robot}

\author{Marco Frego, MIchele Focchi, 
	Luigi Palopoli, .....
%Alessandro Antonucci$^{1}$, Placido Falqueto$^{1}$,
%	Luigi Palopoli$^{1}$, Daniele Fontanelli$^{2}$ %\vspace{0.1cm}
%\IEEEoverridecommandlockouts
% \IEEEpubid{\makebox[\columnwidth]
% 	{\hfill : 978-1-5090-6299-7/17/\$31.00~\copyright~2017 European Union}
%	\hspace{\columnsep}\makebox[\columnwidth]{ }}
%\thanks{$^{1}$A. Antonucci,	P. Falqueto, and L. Palopoli
%    are with the Department of Information Engineering and Computer Science (DISI), University of Trento, Via Sommarive 9, Trento, Italy (e-mail:
%		\{alessandro.antonucci, placido.falqueto, luigi.palopoli\}@unitn.it) }%
%\thanks{$^{2}$D. Fontanelli is with the Department
%	of Industrial Engineering (DII), University of Trento, Via
%	Sommarive 9, Trento, Italy (e-mail:
%	daniele.fontanelli@unitn.it) }
}

\begin{document}

\maketitle
\thispagestyle{empty}
\pagestyle{empty}



\begin{abstract}
	
\end{abstract}

%%%%%%%%%%%%%%%%%%%%%%%%%%%%%%%%%%%%%%%%%%%%%%%%%%%%%%%%%%%%%%%%%%%%%%%%%%%%%%%%

\section{Model}
Consider a spherical pendulum shown as in figure \ref{fig:sp} .
\begin{figure}
\includegraphics[width=\columnwidth]{figs/spherical_pendulum.png}
\caption{Spherical pendulum}
\label{fig:sp}
\end{figure}
We assume that:
\begin{enumerate}
\item The friction can be neglected
\item The mass is entirely concentrated in the body attached to the wire
\item The wire can be unwounded and rewounded but it is rigid and remains completely elongated (i.e., it cannot make bends)
\item The rewinder can pull the wire when winding it but cannot push it (i.e., it can only act as a brake during the unwinding phase)
\item The system is acted on by the following forces:
  \begin{itemize}
  \item The mass weight
  \item The rewinder pull action or braking action $\mathbf{F}_r$ oriented along the wire
  \item An impulsive push force $\mathbf{F}_u$ that the robot can generate when it is attached to the mountain wall
  \end{itemize}
\end{enumerate}
Let $e$ be a frame attached to the mass and with axis oriented along the $x$ axis and let $w$ be the world frame attached to
the suspension point of the pendulum.
If we consider the homogeneous transformation linking $e$ to $w$, we can write:
\[
  T_e^w = \begin{bmatrix} R_z(\phi) &\begin{matrix}0\\0\\0 \end{matrix}  \\ \begin{matrix} 0 &0 &0 \end{matrix} & 1\end{bmatrix}  \begin{bmatrix} R_y(\pi/2 - \theta) &\begin{matrix}0\\0\\0 \end{matrix}  \\ \begin{matrix} 0 &0 &0 \end{matrix} & 1\end{bmatrix}  \begin{bmatrix} I_{3,3} &\begin{matrix}0\\0\\0 \end{matrix}  \\ \begin{matrix} L &0 &0 \end{matrix} & 1\end{bmatrix}  
\]
where $R_z(\phi)$ is the rotation matrix of $\phi$ around $z$ and $R_y(\alpha)$ is the rotation matrix of $\alpha$ around $y$.
Overall, we have:
\[
  T_e^w = \begin{bmatrix} c_\phi s_\theta & -s_\phi & c_\phi c_\theta & l c_\phi s_\theta\\
    s_\phi s_\theta & c_\phi & c_\theta s_\phi & l s_\phi s_\theta\\
  -c_\theta & 0 & s_\theta &-l c_\theta\\
  0 & 0 &0 &1\end{bmatrix}
\]
where $c_x$ is a shorthand for $\cos x$ and $s_x$ is a shorthand for $\sin x$.
From this equation, we have that the position $\mathbf{p}$ of the mass is given by:
\begin{align*}
  \mathbf{p} = \begin{bmatrix} x\\ y\\ z \end{bmatrix} =
  \begin{bmatrix} l s_\theta c_\phi\\
   l s_\theta s_\phi\\
  -l c_\theta \end{bmatrix}
\end{align*}
From this equation it is possible to compute the velocity of the mass and hence the kinetic $T$ and the potential energy $V$:
\begin{align*}
  T &=\frac{ m v^2}{2} = \\
    &= \frac{m}{2} \left(\dot{x}^2 + \dot{y}^2 + \dot{z}^2\right) = \\
    &=  \frac{m}{2} l^2 \left(\dot{\theta}^2 + s_\theta^2 \dot{\phi}^2\right) + \frac{m}{2}\dot{l}^2\\
  V &= mgz\\
  &= -mgl c_\theta
\end{align*}
    The Lagrangian function is given by:
    \begin{equation}
      L = T - V = \frac{m}{2}\dot{l}^2+\frac{m}{2} l^2 \left(\dot{\theta}^2 + s_\theta^2 \dot{\phi}^2\right) + mgl c_\theta
    \end{equation}
    The generalized coordinates are in this case given by $q_1 = \theta, q_2 = \phi,$ and $q_3 = l$ .
    If we apply the d'Alembert principle, the Lagrangian Equation can be written as:
    \[
      \frac{d}{dt}\frac{\partial L}{\partial \dot{q}_i} - \frac{\partial L}{\partial q_i} = Q_i^p ,
      \]
      where $Q_i^p$ is the generalized force $Q_i^p = (\mathbf{F}_r + \mathbf{F}_u) \cdot \frac{\partial \mathbf{p}}{\partial q_i}$.
      Observe that  $\mathbf{F}_r$ oriented along the rope and $\mathbf{F}_u$ does not have any component along the rope (otherwise
       it will forms bends):
      \begin{align*}
        \mathbf{F}_r = F_r \begin{bmatrix}
          c_\phi s_\theta\\
          s_\phi s_\theta\\
          -c_\theta
        \end{bmatrix}^T = F_r \mathbf{f}_r\\
        \mathbf{F}_u = F_{u,t}\begin{bmatrix}
          -s_\phi\\
          c_\phi\\
            0
          \end{bmatrix}^T + F_{u,n} \begin{bmatrix}
          c_\phi c_\theta \\
          s_\phi c_\theta\\
            s_\theta
          \end{bmatrix}^T = F_{u,t} \mathbf{f}_{u,t} + F_{u,n} \mathbf{f}_{u,n},   
      \end{align*}  
      while
      \begin{align*}
        \frac{\partial \mathbf{p}}{\partial q_1} = \frac{\partial \mathbf{p}}{\partial \theta} &= \begin{bmatrix}
          l c_\phi c_\theta\\
          l s_\phi c_\theta\\
          l s_\theta
        \end{bmatrix}\\
        \frac{\partial \mathbf{p}}{\partial q_2} = \frac{\partial \mathbf{p}}{\partial \phi} &= \begin{bmatrix}
          - l s_\phi s_\theta\\
          l c_\phi s_\theta\\
          0
        \end{bmatrix}\\
        \frac{\partial \mathbf{p}}{\partial q_3} = \frac{\partial \mathbf{p}}{\partial l} &= \begin{bmatrix}
          c_\phi s_\theta\\
          s_\phi s_\theta\\
          -c_\theta
        \end{bmatrix}\\
      \end{align*}
      Hence,
       \begin{align*}
         Q_1^p &= \left(\mathbf{F}_r + \mathbf{F}_u\right)   \frac{\partial \mathbf{p}}{\partial q_1}\\
               &= F_r \mathbf{f}_r \cdot  \frac{\partial \mathbf{p}}{\partial \theta} + \\
               &+ F_{u,t} \mathbf{f}_{u,t} \cdot  \frac{\partial \mathbf{p}}{\partial \theta} + \\
               &+ F_{u,n} \mathbf{f}_{u,n} \cdot  \frac{\partial \mathbf{p}}{\partial \theta}  =  \\
               &= F_{u,n} \mathbf{f}_{u,n} \cdot  \frac{\partial \mathbf{p}}{\partial \theta} = \\
         &= F_{u,n} l ,
       \end{align*}
       \begin{align*}
         Q_2^p &= \left(\mathbf{F}_r + \mathbf{F}_u\right)   \frac{\partial \mathbf{p}}{\partial q_2}\\
               &= F_r \mathbf{f}_r \cdot  \frac{\partial \mathbf{p}}{\partial \phi} + \\
               &+ F_{u,t} \mathbf{f}_{u,t} \cdot  \frac{\partial \mathbf{p}}{\partial \phi} + \\
               &+ F_{u,n} \mathbf{f}_{u,n} \cdot  \frac{\partial \mathbf{p}}{\partial \phi}   = \\
               &= F_{u,t} \mathbf{f}_{u,t} \cdot  \frac{\partial \mathbf{p}}{\partial \phi} =\\
         &= F_{u,t} l s_\theta,
       \end{align*}
              \begin{align*}
         Q_3^p &= \left(\mathbf{F}_r + \mathbf{F}_u\right)   \frac{\partial \mathbf{p}}{\partial q_3}\\
               &= F_r \mathbf{f}_r \cdot  \frac{\partial \mathbf{p}}{\partial l} + \\
               &+ F_{u,t} \mathbf{f}_{u,t} \cdot  \frac{\partial \mathbf{p}}{\partial l} + \\
               &+ F_{u,n} \mathbf{f}_{u,n} \cdot  \frac{\partial \mathbf{p}}{\partial l}   = \\
               &= F_r \mathbf{f}_r \cdot  \frac{\partial \mathbf{p}}{\partial l} =\\
         &= F_{r} .
       \end{align*}

      
      
      The first Lagrangian equation yields
      \begin{align*}
        &\frac{d}{dt}\left(\frac{\partial L}{\partial \dot{\theta}} \right) - \frac{\partial L}{\partial \theta} = Q_1^p    \\
        &\frac{d}{dt}\left(m l^2 \dot{\theta}\right) - (m l^2 s_\theta c_\theta \dot{\phi}^2 - m g l s_\theta) = F_{u,n} . l \\ 
        &  m l^2 \ddot{\theta} + 2 m l \dot{\theta} \dot{l} - m l^2 s_\theta c_\theta \dot{\phi}^2 + m g l s_\theta = F_{u,n} . l\\
        & \ddot{\theta} + \frac{2}{l} \dot{\theta} \dot{l} - c_\theta s_\theta \dot{\phi}^2 + \frac{g}{l}  s_\theta =  \frac{F_{u,n}  }{ml}    
      \end{align*}

      The second Lagrangian equation yields
   \begin{align*}
	&\frac{d}{dt}\left(\frac{\partial L}{\partial \dot{\phi}} \right) - \frac{\partial L}{\partial \phi} = Q_2^p    \\
	&\frac{d}{dt}\left(m l^2 s_\theta^2 \dot{\phi}\right) - 0 = F_{u,t} . l . s_\theta\\
	& \ddot{\phi} ml^2 s^2_\theta + 2 m l \dot{l} s_\theta^2 \dot{\phi} + 2 m l^2 s_\theta c_\theta \dot{\theta} \dot{\phi}  = F_{u,t} . l . s_\theta\\
	&\ddot{\phi} + 2 \frac{c_\theta}{s_\theta} \dot{\theta} \dot{\phi} + \frac{2}{l} \dot{\phi} \dot{l} = \frac{F_{u,t}}{ml s_\theta}
   \end{align*}
     
      The third Lagrangian equation is as follows:
      \begin{align*}
        &\frac{d}{dt}\left(\frac{\partial L}{\partial \dot{l}} \right) - \frac{\partial L}{\partial l} = Q_3^p    \\
        &\frac{d}{dt}\left(m\dot{l}\right) - (m l \dot{\theta}^2 + m l s^2_\theta \dot{\phi}^2 + m g c_\theta) = F_r \\
        &m \ddot{l} - m l \dot{\theta}^2 - m l s^2_\theta \dot{\phi}^2 - m g c_\theta = F_r \\
        &\ddot{l} - l \dot{\theta}^2 - l s^2_\theta \dot{\phi}^2 - g c_\theta = \frac{F_r}{m} 
      \end{align*}
      Overall, the nonlinear equations of the system are:
      \begin{align}
         &\ddot{\theta} + \frac{2}{l} \dot{\theta} \dot{l} - c_\theta s_\theta  \dot{\phi}^2 + \frac{g}{l}  s_\theta =  \frac{1}{ml}F_{u,n} \\
         &\ddot{\phi} + 2 \frac{c_\theta}{s_\theta} \dot{\theta} \dot{\phi} + \frac{2}{l} \dot{\phi} \dot{l} = \frac{1}{ml s_\theta} F_{u,t} \\
         &\ddot{l} - l \dot{\theta}^2 - l s^2_\theta \dot{\phi}^2 - g c_\theta = \frac{1}{m} F_r 
      \end{align}


\section{System in canonical form}
The system derived from the physical model is
%%%
\begin{EQ}[rcl]
\ddot{\theta}+2\dot{\theta}\dot{l}-l\dot{\phi}^2\sin\theta\cos\theta+g\sin\theta &=& \frac{F_{u,n}}{m},\\
\ddot{\phi}l\sin^2\theta+2\dot{\phi}\left(\dot{l}\sin^2\theta+l\dot{\theta}\sin\theta\cos\theta\right)&=& \frac{F_{u,t}\sin\theta}{m},\\
\ddot{l}+l\left(\dot{\theta}^2+\dot{\phi}^2\sin^2\theta\right)+g\cos\theta&=&\frac{F_{r}}{m}.
\end{EQ}

%%%%
%\begin{EQ}[rcl]
%\ddot{\theta}(t)+2\dot{\theta}(t)\dot{l}(t)-l(t)\sin\theta(t)\cos\theta(t)\dot{\phi}^2(t)+g\sin\theta(t) &=& \frac{F_{u,n}(t)}{m},\\
%\ddot{\phi}(t)l(t)\sin^2\theta(t)+2\dot{\phi}(t)\left(\dot{l}(t)\sin^2\theta(t)+l(t)\sin\theta(t)\cos\theta(t)\dot{\theta}(t)\right)&=& \frac{F_{u,t}(t)\sin\theta(t)}{m},\\
%\ddot{l}(t)+l(t)\left(\dot{\theta}^2(t)+\sin^2\theta(t)\dot{\phi}^2(t)\right)+g\cos\theta(t)&=&\frac{F_{r}(t)}{m}.
%\end{EQ}


Recast as a first order system with the substitutions
%%%
\begin{EQ}
\dot{\theta} = \omega,\quad \dot{\phi}=\psi,\quad \dot{l}=r.
\end{EQ}
%%%
Reorder the system in canonical form $\dot{\bm{x}}(t)=f(\bm{x}(t),\bm{u}(t))$:
%%
\begin{EQ}[rcl]
\dot{\theta} &=& \omega,\\
\dot{\phi}&=&\psi,\\
\dot{l}&=&r,\\
\dot{\omega} &=& -2\omega r +2l\psi^2\sin\theta\cos\theta -g\sin\theta+\frac{F_{u,n}}{m},\\
\dot{\psi} &=& \frac{-2\psi\left(r\sin^2\theta+l\omega\sin\theta\cos\theta\right)+\frac{F_{u,t}(t)\sin\theta}{m}}{l\sin^2\theta},\\
\dot{r} &=& -l\left(\omega^2+\psi^2\sin^2\theta\right)-g\cos\theta +\frac{F_{r}}{m}.
\end{EQ}
%%%
The boundary conditions for a movement starting and ending at rest ($v=0$) are
%%%
\begin{EQ}[rclcl]
x &=& l\sin\theta\cos\phi &=& x_0\\
y &=& l\sin\theta\sin\phi &=& y_0\\
z &=& -l\cos\theta        &=& z_0\\
x &=& l\sin\theta\cos\phi &=& x_T\\
y &=& l\sin\theta\sin\phi &=& y_T\\
z &=& -l\cos\theta        &=& z_T\\
\end{EQ}
%%%
from which,
%%%
\begin{EQ}
l = \sqrt{x_0^2+y_0^2+z_0^2}, \quad \theta= \atandue\left(-z_0,\sqrt{x_0^2+y_0^2}\right), \quad \phi = \atandue(y_0,x_0).
\end{EQ}

For the velocity, at $t=0$:

\begin{EQ}[rclcl]
\dot{x} &=& (\cos(\theta)l\omega+r\sin(\theta))\cos(\phi)-l\sin(\theta)\psi\sin(\phi) &=& 0\\
\dot{y} &=& (\cos(\theta)l\omega+r\sin(\theta))\sin(\phi)+l\sin(\theta)\psi\cos(\phi) &=& 0\\
\dot{z} &=& -r\cos(\theta)+l\omega\sin(\theta) &=& 0
\end{EQ}

At $t=T$:
\begin{EQ}[rclcl]
\dot{x} &=& r\sin(\theta)\cos(\phi)+l\omega\cos(\theta)\cos(\phi)-l\sin(\theta)\psi\sin(\phi) &=& 0\\
\dot{y} &=& r\sin(\theta)\sin(\phi)+l\omega\cos(\theta)\sin(\phi)+l\sin(\theta)\psi\cos(\phi) &=& 0\\
\dot{z} &=& -r\cos(\theta)+l\omega\sin(\theta)  &=& 0.
\end{EQ}

Sufficient conditions to have zero velocity at $t=0$ and at $t=T$ are
%%%
\begin{EQ}
\dot{l}=r=0, \qquad \dot{\theta}=\omega=0, \quad \dot{\phi}=\psi = 0.
\end{EQ}
%%%
      
      %\bibliographystyle{IEEEtran}
%\bibliography{biblio}

%%%%%%%%%%%%%%%%%%%%%%%%%%%%%%%%%%%%%%%%%%%%%%%%%%%%%%%%%%%%%%%%%%%%%%%%%%%%%%%%

% that's all folks
\end{document}