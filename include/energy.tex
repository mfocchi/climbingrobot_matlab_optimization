In this section we show how energy considerations can be used to set
up a motion planning algorithm, which can be seen as a means to find
reasonable feasible solutions that could be used in their own right or
as a hint to kick.start the execution of optinmal control agorithms.

Our ideas relies on the simple physical principle known as
"conservation of energy". Inosfar as the forces applied to the system
are conservative, the total energy (i.e., the sum of the kinetic and
of the potential energy) remains constant. In our case, the system is
acted on by the following forces: the gravity force, the impulsive
force $\mathbf{F}_u$ and the winding/unwinding force
$\mathbf{F}_r$. Whilst the gravity force is obviously conservative,
the same cannot be said in general for $\mathbf{F}_u$ and for $\mathbf{F}_r$.
However, due to its impulsive nature, $\mathbf{F}_u$ can be accounted
for as an instantaneous reset of the angular velocities, which can be
computed as a function of the impulse strength using standard
mathematical tools~\cite{dim17}. As regards, the winding/unwinding force $\mathbf{F}_r$,
we can impose a conservative force behaviour by constraining $\mathbf{F}_r$ to be generated
by an elastic law $F_r = k(l-l_0)$, where $k$ and $l_0$ have to be suitably identified
in order to solve the motion task of interest.

Under this choice, the total energy of the system is given by:
\begin{equation}
\begin{aligned}
&T = \frac{m l^2}{2} \left(\dot{\theta}^2 + s_\theta^2 \dot{\phi}^2 +\dot{l}^2 \right) \\
&U = -mglc_\theta + \frac{k}{2}(l-l_0)^2 \\
&E = U + T    .
\label{eq:energy}
\end{aligned}
\end{equation}
Energy is a function of time $E(t)$, which is constrained to remain constant for the conservative
nature of the forces applied to the system: $\forall t_1 > 0, t_2 >0 \,\,E(t_1) = E(t_2)$.
Ecluding the instant $t = 0$ from energy conservation allows us to discharge the effect of the non conservative term $\mathbf{F}_u$,
whose only effect is to reset the initial velocities and hence the initial kinetic energy.

With these consideration in mind, our motion planning problem can be
approached by looking for the iso-energetic curve that joins the
initial point $(x_0, y_0, z_0)$ with the final point $(x_f, y_f,
z_f)$.  Clearly, by imposing the initial point, we can find an
infinite set of iso-energetic trajectories parametrised by different
values of the initial velocities $\dot{\theta}^2(0)$,
$\dot{\phi}^2(0)$ (which in turn can be found as a function of
$F_{u,t}$ and $F_{u,n}$) and of the eleastic force parameters $k$ and
$l_0$).
Within this set, we can identify the curves that optimise some
functions of interest (e.g., the time to reach the destination).  
\lui{Here we should find analytical conditions for the existence of a solution. I think
it is possible by requiring a constant enery at the two boundaries. We get one equation and one unknowns. The problem is that
the energy depends linearly on the kinetic terms at the boundaries (6 variables) and on $k$, while it has a non linear term $k l_0$ }.

\subsection{Approximate solution}
The method outlined above still requires the solution of a differential equation. We can look for approximate methods.
The idea is to look for approximate polynomial functions for the solutions of the differential equation. The coefficients of these functions can be found by requiring that the
energy evaluated at different points remains "as constant as possible".
We found it particularly convenient to use cubic polynoimials to approximate the functions determining the evolution of the trajectories:
\begin{equation}
\begin{aligned}
\cos \theta(t) &\approx p_a(t) = a_3 t^3 + a_2 t^2 + a_1 t + a_0\\
\cos \phi(t) &\approx p_b(t) = b_3 t^3 + b_2 t^2 + b_1 t + b_0\\
l(t) &\approx p_c(t) = c_3 t^3 + c_2 t^2 + c_1 t + c_0 .
\end{aligned}
\label{eq:cubic}
\end{equation}

\subsubsection{Constraints}

By using this approximation we can set up the following constraints.

\paragraph{Initial and final position}
When time equals $0$, we have:

\begin{equation}
\begin{aligned}
&l(0) = \sqrt{x_0^2 + y_0^2 + z_0^2} &\rightarrow& c_0 = \sqrt{x_0^2 + y_0^2 + z_0^2}\\
&-l(0) \cos \theta(0) = z_0 &\rightarrow& a_0 = -\frac{z_0}{c_0}\\
&l(0) \sin \theta(0) \cos \phi(0) = x_0 &\rightarrow& b_0 = \frac{x_0}{c_0 \sqrt{1-a_0^2}}
\end{aligned}
\end{equation}

At final time $t_f$, which is an unknown, we will have:
\begin{equation}
\begin{aligned}
&l(t_f) =  \sqrt{x_f^2 + y_f^2 + z_f^2} &\rightarrow& p_c(t_f) = \sqrt{x_f^2 + y_f^2 + z_f^2}\\
&-l(t_f) \cos \theta(f) = z_f &\rightarrow & p_c(t_f) p_a(t_f) = z_f .
\end{aligned}
\end{equation}

\paragraph{Energy invariance}
Plugging Equation~\ref{eq:cubic} into Equation~\ref{eq:energy}, we obtain energy as a function of time
parametrised by the coeeficients of the cubics $E_{a,\,b,\,c}(t)$.
Splitting the segment $[0, t_f]$ into $N$ intervals of equal length, we can impose the relation  
\begin{equation}
\begin{aligned}
 E_{a,\,b,\,c}(0) &= E_{a,\,b,\,c}(\frac{t_f}{N})\\
 E_{a,\,b,\,c}(\frac{t_f}{N}) &= E_{a,\,b,\,c}(2\frac{t_f}{N})\\
 \ldots\\
 E_{a,\,b,\,c}((N-1)\frac{t_f}{N}) &= E_{a,\,b,\,c}(t_f)\\
\end{aligned}
\label{eq:constraintEnergy}
\end{equation}


\paragraph{Trigonometric consistency}
Clearly, we will have to impose the condition:
\begin{align}
\forall t :  \left\| p_a(t) \right\| \leq 1\\
\forall t :  \left\|p_b(t) \right\| \leq 1 .
\end{align}

Observing that a generic cubic equation (say $p_a(t)$)  can have zero or two maximum minimum poinnts, it is sufficient to impose these constraints at
\begin{equation}
t = \begin{cases}
\left\{0, t_f\right\} & \text{if } a_2^2 - 3 a_1 a_3 < 0\\
\left\{0, t_f, \frac{-a_2 \pm \sqrt{a_2^2 - 3 a_1 a_3}}{3 a_3} \right\} & \text{if } a_2^2 - 3 a_1 a_3 < 0\\
\end{cases}
\end{equation}

In essence, we can solve the problem under each of the two said assumptions on the sign of $a_2^2 - 3 a_1 a_3$ and retain the best solution.

\subsection{Cost function}
The cost function can be chosen in different ways:
\begin{enumerate}
\item $\min t_f$ if we aim to minimise time,
\item $\min T(t_f)$ if we aim to minime the residual kinetic energy which will hve to be dissipated when the robot lands,
\item $\min \lambda_1 t_f + \lambda_2 T(t_f)$ where $\lambda_1, \lambda_2$ are weighting factors that allow us to seek different tradeoff solutions .
\end{enumerate}

In addition, it is possible to "soften" the constraint on energy introducing a slack variable $\Delta$. In this case the constraint ~\eqref{eq:constraintEnergy} becomes
\begin{equation}
\begin{aligned}
 \left \| E_{a,\,b,\,c}(0) - E_{a,\,b,\,c}(\frac{t_f}{N}) \right \| \leq \Delta\\
 \left \| E_{a,\,b,\,c}(\frac{t_f}{N}) - E_{a,\,b,\,c}(2\frac{t_f}{N}) \right \| \leq \Delta \\
 \ldots\\
  \left \| E_{a,\,b,\,c}((N-1)\frac{t_f}{N}) - E_{a,\,b,\,c}(t_f) \right \| \leq \Delta .
\end{aligned}
\label{eq:constraintEnergy1}
\end{equation}

The salck variable can become part of the cost function. For instance, we can set the cost function as $\min \lambda_1 t_f + \lambda_2 T(t_f) + \lambda_3 \Delta$.
By increasing the value of $\lambda_3$, we increase the importance of respecting Energy Invariance.


\subsection{Reconstructing impulsive forces}
Once the optimisation problem outliend above has been solved, we find a reference trajectory (given by the cubic expressions $p_a(t)$, $p_b(t)$, $p_c(t)$).
In particular, $\dot{p}_a(0)$ and $\dot{p}_b(0)$ give us the vlaue of the initial velocities $\dot{\theta}(0^+)$ and $\dot{\phi}(0^+)$ respectively.
Using the technique shown in ~\cite{dim17}, it is possible to find the relation between the velocity "jumps" $\dot{\theta}(0^-) \rightarrow \dot{\theta}(0^+)$ and
$\dot{\phi}(0^-) \rightarrow \dot{\phi}(0^+)$.
In particular, we have:
\begin{equation}
\begin{aligned}
  \dot{\theta}(0^+) = \dot{\theta}(0^-) + \frac{1}{m l(0)} F_{u,n}\\
  \dot{\phi}(0^+) = \dot{\phi}(0^-) + \frac{1}{m l(0) \sin \theta(0)} F_{u,t} .
  \end{aligned}
\end{equation}
Such equations can be used to impose additional constraint on force, e.g., the friction cone, right into the
optimisation problem.